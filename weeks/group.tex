\chapter{Groups}
\section{Basic Definitions}
\begin{definition}
Let $S$ be a set. A \textit{binary operation} on $S$ is a map:
\[
* : S \times S \to S
\]
Let $T \subseteq S$ be a subset. We say that the binary operation is \textit{closed} in $T$ if:
\[
\forall a, b \in T, \quad a * b \in T
\]
\end{definition}

\begin{example} Let $S = \mathbb{Z}$. Then the following are binary operations:
\begin{itemize}
    \item $S = \mathbb{Z}, \quad * = +$ (addition)  
    \item $S = \mathbb{Z}, \quad * = \times$ (multiplication)
\end{itemize}
Let 
\[T = \{\text{all even integers}\} \subset \mathbb{Z}.
\] 
For $a, b \in T$, $a+b \in T$ (the sum of two even numbers is even).  
Hence $(T, +)$ is closed in $(S, +)$. Also, if $a,b \in T$, $a \cdot b \in T$ (the product of two even numbers is even).  
Therefore, $(T, \times)$ is also closed in $(S, \times)$.

However, let 
\[
T' = \{\text{all odd integers}\}
\]
Then $(T', +)$ is \textbf{not} closed in $(S, +)$. But for any $(2p+1), (2q+1) \in T'$:
\[
(2p+1)\cdot(2q+1) = 4pq + 2p + 2q + 1
= 2(2pq + p + q) + 1 \in T'
\]
Therefore, $(T', \times)$ is closed in $(S, \times)$.
\end{example}

\begin{example}
\begin{enumerate}
\item Let $S = \mathbb{R}^n$, and $* =$ be the vector addition operation.  
If $W \leq \mathbb{R}^n$ is a vector subspace, then $(W, +)$ is closed in $(\mathbb{R}^n, +)$.

\item Let $S = M_{n\times n}(\mathbb{R})$, and $* =$ multiplication of the ne matrix. Suppose $T = GL_n(\mathbb{R})$ is the subset of all invertible real matrices.  
To check if $T$ is closed in $(S, \times)$:
If $A, B \in T$ (invertible matrices), then  
$A \cdot B$ is invertible since
\[
(AB)^{-1} = B^{-1}A^{-1}
\]
or alternatively:
\[
\det(AB) = \det(A)\det(B) \neq 0
\]
Hence, $T$ is closed in $(S, \times)$.
\end{enumerate}
\end{example}

\begin{definition} [Group] \label{def:group}
A \textbf{group} $G$ is a set along with a binary operation $* : G\times G \to G$ satisfying:
\begin{enumerate}
\item \((a*b)*c = a * (b * c) \quad \forall a,b,c \in G\) \quad (associativity)
\item \(\exists \ e \in G \ \text{such that} \ e * a = a * e = a, \ \forall a \in G\) \quad (identity)
\item \(\forall a \in G, \ \exists b \in G \ \text{s.t.} \ a * b = b * a = e\)  
($b$ is often written as $a^{-1}$, called the inverse of $a$)
\end{enumerate}
\end{definition}
\begin{example}
\begin{itemize}
\item[(a)] $(\mathbb{Z}, +)$ is a group. To see this, note that
\[
\begin{cases}
(a + b) + c = a + (b + c) \quad  \\
0 + a = a + 0 = a \quad \forall a \in \mathbb{Z} \quad (\text{i.e., $e = 0$}) \quad \\
a + (-a) = (-a) + a = 0 \quad (\text{i.e., $a^{-1} = -a$})
\end{cases}
\]
\item[(b)] Similarly $(\mathbb{Q}, +), (\mathbb{R}, +), (\mathbb{C}, +)$ as well as $(\mathbb{Z}_n,+)$ are groups.
\item[(c)] $(\mathbb{Z}, -)$ is \textbf{not} a group, since
\((a - b) - c \neq a - (b - c)\) and \(0 - a \neq a - 0 = a\)
in general, i.e. it is not associative.
\item[(d)] $(\mathbb{Z}, \times)$ is also \textbf{not} a group. Although $(a \times b) \times c = a \times (b \times c)$ is associative, and $a \times 1 = 1 \times a = a$ (so that the identity can be taken as $e = 1$), but there is {\it no} $a \in \mathbb{Z}$ such that
$2 \times a = 1 = e$.)
\item[(e)] $(\mathbb{Q}, \times)$ is \textbf{not} a group. Although now one can take $a = \frac{1}{2}$ such that $2 \times \frac{1}{2} = 1$, yet the element $0 \in \mathbb{Q}$ does not have a multiplicative inverse.
\item[(f)] To resolve the issue, let $\mathbb{Q}^* = \{ q \in \mathbb{Q} \mid q \neq 0 \}$. Then $(\mathbb{Q}^*, \times)$ is a group.
\item[(g)] Similarly, $(\mathbb{R}^*, \times)$ and $(\mathbb{C}^*, \times)$ are groups.
\item[(h)] The set of $2 \times 2$ real matrices, denoted $M_2(\mathbb{R})$, under matrix multiplication $(M_2(\mathbb{R}), \times)$ is \textbf{not} a group, because not all matrices have a multiplicative inverse (namely, singular matrices do not have a multiplicative inverse).
\item[(i)] Let $GL(2, \mathbb{R}) = \{ A \in M_2(\mathbb{R}) \mid \det(A) \neq 0 \}$ be the set of $2 \times 2$ invertible matrices with real entries. Then $(GL(2, \mathbb{R}), \times)$ is a group, known as the general linear group of degree 2 over $\mathbb{R}$.
\end{itemize}
\end{example}

\begin{question}
Given $n \in \mathbb{N}$, for which $a \in \mathbb{Z}$ does $[a]_n$ (the congruence class of $a$ modulo $n$) have a multiplicative inverse in $\mathbb{Z}_n$?
\end{question}

\noindent\textbf{Answer:} 
An element $[a]_n \in \mathbb{Z}_n$ has a multiplicative inverse if and only if $\gcd(a, n) = 1$. 
%{\color{red} add proof using Bezout's theorem. Use \texttt{label} $\backslash \{\ \}$ and \texttt{ref} $\backslash \{\ \}$ to refer to the previously stated theorem}
For instance, when $n=6$, then:
\begin{itemize}
    \item For $a=3$, $\gcd(3, 6) = 3 \neq 1$. Therefore, $[3]_6$ does not have a multiplicative inverse in $\mathbb{Z}_6$.
    \item For $a=5$, $\gcd(5, 6) = 1$. Therefore, $[5]_6$ has a multiplicative inverse in $\mathbb{Z}_6$. Indeed, $[5]_6 \cdot [5]_6 = [25]_6 \equiv [1]_6 \pmod{6}$, so $[5]_6^{-1} = [5]_6$.
\end{itemize}

\begin{remark}
When there is no ambiguity, we often write $ab$ instead of $a \ast b$. Also, for all $m \in \mathbb{Z}$, we denote
    $$a^m := \begin{cases}
        \overbrace{a \ast a \ast \dots \ast a}^{m\ times} &\text{if}\ m > 0 \\
        e &\text{if}\ m = 0 \\
        \overbrace{a^{-1} \ast a^{-1} \ast \dots \ast a^{-1}}^{-m\ times} &\text{if}\ m < 0
    \end{cases}$$

In the case when the operation $\ast$ is the `usual' addition $+$, we call $(G,+)$ {\bf additive group}, and write
$$0 := e, \quad -a := a^{-1}, \quad m \cdot a := a^m.$$
\end{remark}


The order of multiplication matters, namely for $b \in G$, multiplying $a \in G$ on the left or on the right may result in different elements. In the case when they are the same, we have:
\begin{definition}[Abelian Group]
A group $(G, \ast)$ is called \emph{abelian} if its operation is commutative; that is, for all $a, b \in G$, we have $a \ast b = b \ast a$.
\end{definition}

\begin{example}
\begin{itemize}
    \item Examples of abelian groups include: $(\mathbb{Z}, +)$, $(\mathbb{Q}, +)$, $(\mathbb{R}, +)$, $(\mathbb{C}, +)$, and $(\mathbb{Z}_n, +)$.
    \item The group $(GL(2, \mathbb{R}), \times)$ is NOT abelian, as matrix multiplication is generally not commutative.
\end{itemize}
\end{example}

\begin{proposition}
Let $(G, \ast)$ be a group.
\begin{enumerate}
    \item (Uniqueness of Identity) The identity element $e$ in $G$ is unique.
    \item (Uniqueness of Inverses) For each $a \in G$, its inverse $a^{-1}$ is unique.
    \item (Cancellation Laws) For $a, b, c \in G$:
    \begin{itemize}
        \item If $a \ast b = a \ast c$, then $b=c$.
        \item If $b \ast a = c \ast a$, then $b=c$.
    \end{itemize}
\end{enumerate}
\end{proposition}

\section{Cayley Table}

\begin{definition}[Order of a Group]
The \emph{order} of a group $G$, denoted $|G|$, is the number of elements in the group. If $|G| < \infty$, the group is called a \emph{finite group}.
\end{definition}

\begin{definition}[Cayley Table]
For a finite group $(G, \ast)$, a \emph{Cayley table} (or group table) is a square table where the rows and columns are labeled by the elements of the group. The entry in the row corresponding to $a$ and the column corresponding to $b$ is $a \ast b$.
\end{definition}

\begin{example}
The Cayley table for \(\mathbb{Z}_9^*\) is:
\[
\begin{array}{|c||c | c|c|c|c|c|}
\hline
\times & [1]_9 & [2]_9 & [4]_9 & [5]_9 & [7]_9 & [8]_9 \\
\hline \hline
[1]_9 & [1]_9 & [2]_9 & [4]_9 & [5]_9 & [7]_9 & [8]_9 \\
\hline
[2]_9 & [2]_9 & [4]_9 & [8]_9 & [1]_9 & [5]_9 & [7]_9 \\
\hline
[4]_9 & [4]_9 & [8]_9 & [7]_9 & [2]_9 & [1]_9 & [5]_9 \\
\hline
[5]_9 & [5]_9 & [1]_9 & [2]_9 & [7]_9 & [8]_9 & [4]_9 \\
\hline
[7]_9 & [7]_9 & [5]_9 & [1]_9 & [8]_9 & [4]_9 & [2]_9 \\
\hline
[8]_9 & [8]_9 & [7]_9 & [5]_9 & [4]_9 & [2]_9 & [1]_9 \\
\hline
\end{array}
\]
\end{example}


\begin{proposition}
In a Cayley table for a finite group, each element of the group appears exactly once in each row and exactly once in each column. In other words, for all $a \in G$, the sets
$$\{a \ast g\ |\ g \in G\}$$
and 
$$\{g \ast a\ |\ g \in G\}$$
contains all elements of $G$ exactly once, with no repetitions.
\end{proposition}


%\begin{example} Consider a group $G$ with $|G|=4$. For instance, $(\mathbb{Z}_4, +)$ would have a Cayley table where each row and column contains $\{[0]_4, [1]_4, [2]_4, [3]_4\}$ exactly once. \end{example}

\section{Subgroups}

\begin{definition}[Subgroup]
Let $(G, \ast)$ be a group and $H$ be a non-empty subset of $G$. We say that $H$ is a \emph{subgroup} of $G$, denoted $H \le G$, if $(H, \ast)$ is itself a group under the same operation $\ast$ restricted to $H$.
\end{definition}

\begin{proposition}[Subgroup Test]
Let $(G, \ast)$ be a group and $H$ be a non-empty subset of $G$. Then $H$ is a subgroup of $G$ if and only if:
\begin{enumerate}
    \item For all $a, b \in H$, $a \ast b \in H$, i.e. $\ast$ is closed in $H$.
    \item For all $a \in H$, $a^{-1} \in H$ .
\end{enumerate}
Equivalently, one can check the following:
\begin{center}
    For all $a, b \in H$, $a \ast b^{-1} \in H$.
\end{center}
\end{proposition}

\begin{example}
\begin{enumerate}
    \item $(\mathbb{Z}, +) \le (\mathbb{Q}, +) \le (\mathbb{R}, +) \le (\mathbb{C}, +)$.
    \item If $W \leq \mathbb{R}^n$ is a vector subspace, then $(W, +) \leq (\mathbb{R}^n, +)$ is a subgroup.
    \item Let $(G, *) = (\mathbb{Z}, +)$, and $H = \text{all even integers} = 2\mathbb{Z} \subseteq G$. Take $2p, 2q \in H$ ($p, q \in \mathbb{Z}$). Then $2p + 2q = 2(p+q) \in H$ and $-2p = 2(-p) \in H$. Hence, $H \le G$ is a subgroup of $G$.
    More generally, for all $k \in \mathbb{Z}$, $(k\mathbb{Z},+) \le (\mathbb{Z},+)$.
    
    Meanwhile, if $K = \text{all odd integers} \subseteq G$, then $K$ is \textbf{NOT} a subgroup of $G$, since $1, 3 \in K$ but $1 + 3 \notin K$.
    
    \item Let $(G, *) = (GL(n, \mathbb{R}), \cdot)$, and 
    $$SL(n, \mathbb{R}) := \{ A \in G \mid \det(A) = 1 \}$$ 
    Then $SL(n,\mathbb{R}) \leq GL(n, \mathbb{R})$:
    \begin{itemize} 
    \item For all $A, B \in SL(n, \mathbb{R})$,  $\det(AB) = \det(A) \cdot \det(B) = 1 \cdot 1 = 1$. Therefore, $AB \in SL(n, \mathbb{R})$ 
    for all $A, B \in SL(n, \mathbb{R})$ 
    \item For all $A \in SL(n, \mathbb{R})$, $\det(A^{-1}) = \frac{1}{\det(A)} = \frac{1}{1} = 1$. Therefore, $A^{-1} \in SL(n, \mathbb{R})$ as well.
    \end{itemize}
    \item Let $(G, *) = (\mathbb{Z}_8, +)$. Then $ H = \{0, 4\}$ is a subgroup of $G$, while $K = \{0, 3\}$ is \textbf{NOT} a subgroup.
\end{enumerate}
\end{example}

\begin{definition}
Let $(G, *)$ be a group. Then:
    \begin{enumerate}
        \item $(G, *) \le (G, *)$ is a subgroup of $G$. We say all $H \le G$ satisfying $H \neq G$ a {\bf proper subgroup} of $G$.
        \item $\{e\} \le G$ is a subgroup. We say all $H \le G$ with $H \neq \{e\}$ a {\bf nontrivial subgroup} of $G$.
    \end{enumerate}
\end{definition}


\section{Cyclic Groups}

\begin{definition}[Cyclic Subgroup]
Let $(G, *)$ be a group. A {\bf cyclic subgroup} generated by $g \in G$ is the subgroup $$\langle g \rangle := \{ g^m \mid m \in \mathbb{Z} \}.$$
\end{definition}
\noindent (Exercise: Check $\langle g \rangle \le G$ is a subgroup of $G$, i.e.: for any
$g^a, g^b \in \langle g \rangle \quad \begin{cases} g^a * g^b \in \langle g \rangle \\ (g^a)^{-1} \in \langle g \rangle \end{cases}$.)

\begin{example}
Here are some examples of cyclic subgroups:
\begin{enumerate}
    \item Let $(G, *) = (\mathbb{Z}, +)$, then 
    $$\langle 2 \rangle = \{ 2, 2+2, 2+2+2, \dots, 0, (-2), (-2)+(-2), (-2)+(-2)+(-2), \dots \} = 2\mathbb{Z}.$$
    
    \item Let $(G, *) = (GL(2, \mathbb{R}), \cdot)$, then    $$\left\langle \begin{pmatrix} 1 & 1 \\ 0 & 1 \end{pmatrix} \right\rangle = \left\{ \begin{pmatrix} 1 & 1 \\ 0 & 1 \end{pmatrix}^m \middle| m \in \mathbb{Z} \right\} = \left\{ \begin{pmatrix} 1 & m \\ 0 & 1 \end{pmatrix} \middle| m \in \mathbb{Z} \right\}.$$ 
    ( Check: $\begin{pmatrix} 1 & a \\ 0 & 1 \end{pmatrix} \cdot \begin{pmatrix} 1 & b \\ 0 & 1 \end{pmatrix} = \begin{pmatrix} 1 & a+b \\ 0 & 1 \end{pmatrix}$ )
    
    \item Let $(G, *) = (\mathbb{Z}_8, +)$, then
    $$\langle 2 \rangle = \{ 0, 2, 4, 6, \dots \} = \{ 0, 2, 4, 6 \}$$
    $$\langle 3 \rangle = \{ 0, 3, 6, 9, 12, 15, 18, 21, 24, \dots \} = \{ 0, 3, 6, 1, 4, 7, 2, 5, 0, \dots \} = G$$
\end{enumerate}
\end{example}

\begin{definition}[Cyclic Group]
Let $(G, *)$ be a group. We say $G$ is \textbf{cyclic} if there is $g \in G$ such that $\langle g \rangle = G$.
\end{definition}


\begin{example}
\begin{itemize}
    \item $(G, *) = (\mathbb{Z}, +)$ is cyclic, since $G = \langle 1 \rangle$.
    \item $(G, *) = (\mathbb{Z}_8, +)$ is cyclic, since $G = \langle 3 \rangle = \langle 1 \rangle$.
    \item $(G, *) = (GL(2, \mathbb{R}), \cdot)$ is \textbf{NOT} cyclic, since for all $H = \langle g \rangle$, $H$ is countable but $G$ is uncountable. 
    Hence, $H$ can never be equal to $G$.
    
    \item $(G, *) = (\mathbb{Z}_8^*, \cdot) = \{ 1, 3, 5, 7 \}$. Then the cyclic subgroups of $G$ are: \\
    $\langle 1 \rangle = \{ 1 \}$,\\ 
    $\langle 3 \rangle = \{ 3^0=1, 3^1=3, 3^2=1, 3^3=3, \dots \} = \{ 1, 3 \}$;\\
    $\langle 5 \rangle = \{ 1, 5 \}$; \\
    $\langle 7 \rangle = \{ 1, 7 \}$. 
    
    Note that none of them is equal to $G$. So $G$ is \textbf{not} cyclic.
\end{itemize}
\end{example}

\noindent \textbf{Question:} $(G, *)$ group, $H = \langle g \rangle$. What's the order of $H$? For instance, if $G = (\mathbb{Z}_8, +)$, then $ |\langle [2] \rangle| = 4, \ |\langle [3] \rangle| = 8, \ |\langle [4] \rangle| = 2$.
\begin{definition}[Order] 
Let $g \in G$. The \textit{order} of $g$ is equal to the smallest positive integer $k$ such that $g^k = e$. If no such $k$ exists, then we say the order of $g$ is $\infty$.
\end{definition}

\begin{proposition} \label{prop:ordg}
If $\mathrm{ord}(g) = k$, then $|\langle g \rangle| = k$.
\end{proposition}

\begin{example}
\begin{itemize}
    \item $(G, *) = (\mathbb{Z}_8, +)$. $\text{ord}([2]) = 4$ since $$[2]=[2], [2]+[2]=[4], [2]+[2]+[2]=[6], [2]+[2]+[2]+[2]=[8] = [0].$$
    Also, $\text{ord}([3]) = 8$, since 
    $$1 \cdot [3]=[3],\ 2 \cdot [3] =[6],\ 3 \cdot [3]=[9] = [1],\ 4 \cdot [3]=[12] = [4],$$ $$5 \cdot [3]=[15] = [7],\ 6 \cdot [3]=[18] = [2],\ 7 \cdot [3]=[21] = [5],\ 8 \cdot [3] =[24] = [0].$$
    \item Let $(G, *) = (GL(2, \mathbb{R}), \cdot)$. Then 
    $\text{ord}\left( \begin{pmatrix} 1 & 1 \\ 0 & 1 \end{pmatrix} \right) = \infty$, since $$\begin{pmatrix} 1 & m \\ 0 & 1 \end{pmatrix} = \begin{pmatrix} 1 & 1 \\ 0 & 1 \end{pmatrix}^m \neq \begin{pmatrix} 1 & 0 \\ 0 & 1 \end{pmatrix} \quad \forall m \in \mathbb{N}.$$
\end{itemize}
\end{example}
\vspace{1em}
\begin{remark} 
If $|G| < \infty$ is finite, then $\langle g \rangle \le G$ is a finite subgroup with $|\langle g \rangle| \le |G|$. Indeed, all $H \le G$ has $|H| \mid |G|$ (see Section \ref{sec:Lagrange} - Lagrange's Theorem below). \\
e.g.: If $|G|=12$, then there is \textbf{NO} $H \le G$ with $|H|=8$, can only be $1, 2, 3, 4, 6, 12$. \\
Therefore, $\text{ord}(g) \mid |G|$ (take $H = \langle g \rangle$).
\end{remark}

\vspace{1em}
\section{More examples of Groups}
\subsection*{Permutation group / Symmetric group $S_n$}


\begin{definition} 
A \textit{permutation} of $X_n = \{1, 2, \dots, n\}$ is a bijection $\sigma : X_n \to X_n$. \\
\end{definition}
For example, when $n=3$, $\sigma(1)=2, \sigma(2)=3, \sigma(3)=1$ is a permutation.

\begin{definition} 
The \textbf{symmetric / permutation group} of $n$ elements is the collection of all permutations $\sigma : X_n \to X_n$.
\[ S_n := \{ \sigma : X_n \to X_n \mid \sigma \text{ is bijective} \} \]
\end{definition}
For instance, when $n = 3$, one has:
\begin{center}
\usetikzlibrary{arrows.meta, bending}
\begin{tikzpicture}[
    ball/.style={circle, draw=black, thick, minimum size=6mm, font=\small},
    arrow/.style={-{Stealth}, thick, blue!70!black, shorten >=2pt, shorten <=2pt},
    scale=0.8
]

    \newcommand{\drawpermutation}[7]{
        \begin{scope}[shift={(#2, #3)}]
            % Top Row
            \foreach \i in {1,2,3} {
                \node[ball] (T\i) at (\i*1.2, 2) {\i};
            }
            % Bottom Row
            \foreach \i in {1,2,3} {
                \node[ball] (B\i) at (\i*1.2, 0) {\i};
            }
            % Arrows
            \draw[arrow] (T1) -- (B#4);
            \draw[arrow] (T2) -- (B#5);
            \draw[arrow] (T3) -- (B#6);
            % Label
            \node at (2.4, -1) {\textbf{#7}};
        \end{scope}
    }

    % --- ROW 1: Identity and 3-Cycles ---
    \drawpermutation{id}{0}{0}{1}{2}{3}{}
    \drawpermutation{c123}{5}{0}{2}{3}{1}{}
    \drawpermutation{c132}{10}{0}{3}{1}{2}{}

    % --- ROW 2: Transpositions (Swaps) ---
    \drawpermutation{t12}{0}{-5}{2}{1}{3}{}
    \drawpermutation{t23}{5}{-5}{1}{3}{2}{}
    \drawpermutation{t13}{10}{-5}{3}{2}{1}{}
\end{tikzpicture}
\end{center}
Hence $|S_3|=6$. More generally, 
$$|S_n| = n \cdot (n-1) \cdots 1 = n!.$$
\begin{proposition} 
$(S_n, \circ)$ is a group, where $\circ$ is the composition of functions. 
\end{proposition}
\begin{proof} We need to show that the three axioms of groups hold for $(S_n, \circ)$:
\begin{enumerate}
    \item $(\sigma_1 \circ \sigma_2) \circ \sigma_3 = \sigma_1 \circ (\sigma_2 \circ \sigma_3)$ is true, since composition of functions $\circ$ is associative.
    \item Take $e = \text{id} : X_n \to X_n$, $e(i) := i$ for all $1 \leq i \leq n$. Then $\sigma \circ e = e \circ \sigma = \sigma$.
    \item Since $\sigma$ is bijective, then $\sigma^{-1} : X_n \to X_n$ exists with $\sigma \circ \sigma^{-1} = \sigma^{-1} \circ \sigma = e$.
\end{enumerate}
\end{proof}


\begin{remark} \label{rmk-symmetrysn}
Groups are often used to study "symmetry" of objects. For example, $S_n$ can be used to study symmetry of $n$ identical objects. Here is an example of $n=4$:
\begin{center}
\begin{tikzpicture}[
    ball/.style={circle, draw=black, thick, minimum size=1cm, fill=blue!20},
    arrow/.style={-{Stealth}, thick, shorten >=2pt, shorten <=2pt, gray!80}
]

    % --- Top Row (Original Positions) ---
    \node[ball] (T1) at (0, 3) {};
    \node[ball] (T2) at (2, 3) {};
    \node[ball] (T3) at (4, 3) {};
    \node[ball] (T4) at (6, 3) {};
    
    \node[anchor=west] at (7, 3) {Original};

    % --- Bottom Row (Permuted Positions) ---
    \node[ball] (B1) at (0, 0) {};
    \node[ball] (B2) at (2, 0) {};
    \node[ball] (B3) at (4, 0) {};
    \node[ball] (B4) at (6, 0) {};

    \node[anchor=west] at (7, 0) {Permuted};

    % --- Permutation Arrows ---
    % 1 moves to the 3rd spot
    \draw[arrow] (T1) -- (B3);
    % 2 moves to the 1st spot
    \draw[arrow] (T2) -- (B1);
    % 3 moves to the 4th spot
    \draw[arrow] (T3) -- (B4);
    % 4 moves to the 2nd spot
    \draw[arrow] (T4) -- (B2);

\end{tikzpicture}
\end{center}
Although we have carried out a permutation $1 \mapsto 3$, $2 \mapsto 1$, $3 \mapsto 4$, $4 \mapsto 2$, one cannot tell the difference of the objects before and after permutation. 

As another example, to describe (some) symmetries of $\mathbb{R}^n$, we have: 
$$GL(n, \mathbb{R}) = \{ A : \mathbb{R}^n \to \mathbb{R}^n \mid A \text{ is a bijective linear transformation} \}$$
which can be seen as a certain kind of `permutation' on the points in $\mathbb{R}^n$.
\end{remark}

\subsection*{Calculations on $S_n$}
We use cycle notations to denote elements of $S_n$. 

\begin{definition}
A permutation $\sigma \in S_n$ is called a \textbf{$k$-cycle} (or a cycle of length $k$) if there exists a set of $k$ distinct elements $\{a_1, a_2, \dots, a_k\} \subseteq \{1, 2, \dots, n\}$ such that:
\[
\sigma(a_1) = a_2, \quad \sigma(a_2) = a_3, \quad \dots, \quad \sigma(a_{k-1}) = a_k, \quad \sigma(a_k) = a_1
\]
and $\sigma(i) = i$ for all $i \notin \{a_1, a_2, \dots, a_k\}$.

As an example, the element $\sigma \in S_4$ in Remark \ref{rmk-symmetrysn} is a $4$-cycle $\sigma =(1\ 3\ 4\ 2)$.
\end{definition} 

Note that every $\sigma \in S_n$ can be expressed as a product of {\bf disjoint} cycles (possibly of different lengths):
$$\sigma = (a_1 \dots a_k) (b_1 \dots b_l) \cdots (c_1 \dots c_m) := (a_1 \dots a_k) \circ (b_1 \dots b_l) \circ \cdots \circ(c_1 \dots c_m).$$
where the sets $\{a_1, \dots , a_k\}$, $\{b_1, \dots , b_l\}$, $\cdots$, $\{c_1, \dots , c_m\}$ are disjoint subsets of $X_n$. For instance, in $S_8$, the permutation
\begin{center}
$1 \to 3$,\quad $2 \to 5$,\quad $3 \to 1$,\quad $4 \to 6$,\quad $5 \to 2$,\quad $6 \to 8$,\quad $7 \to 7$,\quad $8 \to 4$ 
\vspace{3mm}
\\
\begin{tikzpicture}[
    ball/.style={circle, draw=black, thick, minimum size=0.8cm, fill=white},
    arrow/.style={-{Stealth}, thick, shorten >=2pt, shorten <=2pt}
]

    % --- Nodes for Top Row (Original Indices) ---
    \foreach \i in {1,...,8} {
        \node[ball] (T\i) at (\i*1.5, 3) {\i};
    }
    \node[anchor=west, font=\small\bfseries] at (12.5, 3) {Index $i$};

    % --- Nodes for Bottom Row (Values) ---
    \foreach \i in {1,...,8} {
        \node[ball] (B\i) at (\i*1.5, 0) {\i};
    }
    \node[anchor=west, font=\small\bfseries] at (12.5, 0) {Image $\sigma(i)$};

    % --- Permutation Arrows (Color-coded by Cycle) ---
    
    % Cycle (1 3) - Red
    \draw[arrow, red] (T1) -- (B3);
    \draw[arrow, red] (T3) -- (B1);

    % Cycle (2 5) - Blue
    \draw[arrow, blue] (T2) -- (B5);
    \draw[arrow, blue] (T5) -- (B2);

    % Cycle (4 6 8) - Teal
    \draw[arrow, teal] (T4) -- (B6);
    \draw[arrow, teal] (T6) -- (B8);
    \draw[arrow, teal] (T8) -- (B4);

    % Cycle (7) - Orange
    \draw[arrow, orange] (T7) -- (B7);
\end{tikzpicture}
\end{center}
can be split into:
\begin{itemize}
    \item $1 \to 3$ and $3 \to 1$. This gives the cycle $(1 \, 3)$.
    \item $2 \to 5$ and $5 \to 2$. This gives the cycle $(2 \, 5)$.
    \item $4 \to 6$, $6 \to 8$, and $8 \to 4$. This gives the cycle $(4 \, 6 \, 8)$.
    \item $7 \to 7$. This is a fixed point, giving the 1-cycle $(7)$.
\end{itemize}
Therefore,
\[
\sigma = (1 \, 3)(2 \, 5)(4 \, 6 \, 8)(7) := (1 \, 3) \circ (2 \, 5) \circ (4 \, 6 \, 8) \circ (7)
\]

Here are some examples in calculating products in $S_n$:
\begin{example}
Let $\sigma = (1\,2\,3)$, $\tau = (1\,3\,4\,2)(7\,6)$ in $S_7$. Then:
\begin{itemize}
    \item $\sigma\tau = (1\,2\,3)(1\,3\,4\,2)(7\,6) = (1)(2)(3\,4)(7\,6) = (3\,4)(6\,7)$.
    \item $\tau\sigma = (1\,3\,4\,2)(7\,6)(1\,2\,3) = (1)(2\,4)(3)(6\,7) = (2\,4)(6\,7)$.
\end{itemize}
Therefore, 
\begin{center}
{\bf $\sigma \circ \tau \neq \tau \circ \sigma$ in general, i.e. $S_n$ is not abelian for $n \geq 3$.} 
\end{center}
\end{example}
\vspace{1em}
\begin{remark}
\begin{itemize}
    \item From now on, we'll express $\sigma \in S_n$ using (disjoint) product of $k$-cycles.
    \item If $\alpha$ and $\beta$ are $k$-cycles with disjoint entries, then $\alpha\beta = \beta\alpha$. \\
    (e.g.: $\alpha = (1\,3\,2)$, $\beta = (4\,7) \Rightarrow \alpha\beta = (1\,3\,2)(4\,7) = (4\,7)(1\,3\,2) = \beta\alpha$). \\
    Otherwise, $\alpha\beta \neq \beta\alpha$ in general.
    \item The inverse of a $k$-cycle is given by:
    \[ (i_1\,i_2 \dots i_k)^{-1} = (i_1\,i_k\,i_{k-1} \dots i_2) \]
    \item More generally, for any $\sigma \in S_n$, $\sigma = (i_1 \dots i_k)(j_1 \dots j_\ell) \dots (m_1 \dots m_p)$ where the cycles are disjoint, then:
    \[ \sigma^{-1} = (i_1\,i_k \dots i_2)(j_1\,j_\ell \dots j_2) \dots (m_1\,m_p \dots m_2) \]
\end{itemize}
\end{remark}

\section*{Alternating Group $A_n$}
\begin{definition}[Transposition]
A 2-cycle $\tau = (i\,j) \in S_n$ is called a \textbf{transposition}.
\end{definition}
\vspace{1em}
\begin{proposition}
Every $\sigma \in S_n$ can be expressed as a product of (NOT necessarily disjoint) transpositions.
\end{proposition}

\begin{proof}
$\sigma = (i_1 \dots i_k)(j_1 \dots j_\ell) \dots (m_1 \dots m_p)$ with each $i, j, \dots, m$ cycles are disjoint. Then for $(i_1 \dots i_k)$:
\[ (i_1 \dots i_k) = (i_1\,i_k)(i_1\,i_{k-1}) \dots (i_1\,i_3)(i_1\,i_2) \]
Do the same for $(j_1 \dots j_\ell), \dots, (m_1 \dots m_p)$. Then we're done. \end{proof}

\begin{remark}
The expression of $\sigma \in S_n$ into product of transpositions is \textbf{NOT} unique. For example, $$\sigma = (2\,3) = (2\,3)(2\,3)(2\,3) = (1\,2)(2\,3)(1\,3).$$ 
But the number of transpositions in each expression of $\sigma$ is always \textbf{even} or \textbf{odd}, as we will see below.
\end{remark}

\begin{lemma} \label{lem-idiseven}
If $e = \tau_1 \cdots \tau_k$, where each $\tau_i$ is a transposition. Then $k \equiv 0 \pmod 2$ is even.
\end{lemma}

\begin{proof}
Apply induction on $k$, that is,
\begin{center}
If $e = \tau_1 \cdots \tau_k$, then $k$ is even. \quad \quad (*)
\end{center}
The case of $k=0$ is trivial. Also, it is obvious that the $k \neq 1$, since $e$ cannot be equal to any single transposition $\tau$. So (*) holds for $k=1$. 

Now suppose by induction hypothesis that (*) holds for all expressions of $e = \tau'_1 \cdots \tau'_{k'}$ for $k' \le m$. Consider 
$$e = \tau_1 \cdots \tau_m \tau_{m+1},$$ 
an expression of $e$ with $(m+1)$ transpositions. Let $\tau_{m+1} = (a\,b)$, Then:

\textbf{Case 1:} $\tau_m = (a\,b)$. In this case, $e = \tau_1 \cdots \tau_{m-1}(a\,b)(a\,b) = \tau_1 \cdots \tau_{m-1}$. By induction, $m-1 \equiv 0 \pmod 2$. Hence, $m+1 \equiv 0 \pmod 2$ is even.

\textbf{Case 2:} $\tau_m \neq (a\,b)$. Then $\tau_m = $
$\begin{cases} 
(a\,c) & c \neq b\\
(b\,d) & d \neq a \\
(i\,j)(a\,b) & \{i,j\} \cap \{a,b\} = \emptyset
\end{cases}$. Then one has
$$\tau_m \tau_{m+1} = 
\begin{cases} 
(a\,c)(a\,b) = (a\,b)(b\,c) \\
(b\,d)(a\,b) = (a\,d)(d\,b) \\
(i\,j)(a\,b) = (a\,b)(i\,j) 
\end{cases}$$ 
(where the right hand side in the new expression ($(b\ c)$, $(d\ b)$, $(i\ j)$) have no "$a$"s). Therefore,
$$e = \tau_1 \cdots \tau_{m-1} \tau_m \tau_{m+1} = \tau_1 \cdots \tau_{m-1} (a\,\star)(\ast\,\ast).$$
where $\ast \neq a$. 

Continue the same argument on  $\tau_1 \dots \tau_{m-1}(a\,*)$. If $\tau_{m-1} = (a\,*)$, then $\tau_{m-1}(a\,*)$ goes away. Then one can apply induction as in Case 1 and get the same conclusion. Otherwise, $\tau_{m-1} \neq (a\,*)$, then we use Case 2 to move one more step to the left and get: 
$$e = \tau_1 \cdots \tau_{m-2} (a\, \square)(\ast \,\ast)(\ast\,\ast)$$ 
with $\ast \neq a$. 

We keep moving $a$ to the left, and claim that Case 1 must occur somewhere, so that we can apply the induction argument to conclude $m+1$ is even. Otherwise, we can keep applying Case 2 to `push' $a$ to the leftmost position, and get
$$e = (a\,\triangle) (*\,*) \cdots (*\,*)$$ 
with $\ast \neq a$. But this \textbf{cannot} happen, since the expression on right-hand-side permutes $a \to \triangle$ (note that $(a\ \triangle)$ is the only transposition on the right that moves $a$), which contradicts that fact that it is the identity element. 
\end{proof}

\begin{proposition}
Let $\sigma = \tau_1 \cdots \tau_k = \tau'_1 \cdots \tau'_\ell$ be two expressions of $\sigma$ as product of transpositions. Then $k \equiv \ell \pmod 2$.
\end{proposition}

\begin{proof}
$$e = \sigma^{-1} \sigma = (\tau_1 \cdots \tau_k)^{-1} (\tau'_1 \cdots \tau'_\ell) = \tau_k^{-1} \cdots \tau_1^{-1} \tau'_1 \cdots \tau'_\ell = \tau_k \cdots \tau_1 \tau'_1 \cdots \tau'_\ell.$$
Then by Lemma \ref{lem-idiseven}, $k + \ell \equiv 0 \pmod 2$.
\end{proof}

\begin{definition}
$\sigma \in S_n$ is called an \textbf{even / odd permutation} if $\sigma$ is a product of an even / odd number of transpositions.
\end{definition}

\begin{definition}[Alternating Group]
The \textbf{alternating group} $A_n$ is the subgroup of $S_n$ given by:
\[ A_n := \{ \sigma \in S_n \mid \sigma \text{ is even} \} \]
\end{definition}
\noindent (Check: $A_n \le S_n$. Also, is the set $\{ \sigma \in S_n \mid \sigma \text{ is odd} \}$ a subgroup?)

\begin{proposition}
$|A_n| = \frac{1}{2} |S_n| = \frac{n!}{2}$
\end{proposition}

\begin{proof}
Let $\tau \in S_n$ be a transposition and define $f : A_n \to \{ \sigma \in S_n \mid \sigma \text{ odd} \}$ by:
$$f(\sigma) := \sigma \tau.$$ 
Note that $f$ is bijective with $f^{-1} = f$. Hence, $|A_n| = |\{ \sigma \in S_n \mid \sigma \text{ odd} \}|$. 

Since $A_n \cap \{ \sigma \in S_n \mid \sigma \text{ odd} \} = \emptyset$, and
$A_n \sqcup \{ \sigma \in S_n \mid \sigma \text{ odd} \} = S_n$, therefore $$|A_n| = \frac{|S_n|}{2} = \frac{n!}{2}.$$ 
\end{proof}

\begin{example}

\begin{enumerate}
    \item $A_4$ is the group of ``symmetries of a tetrahedron''.

\begin{center}
\usetikzlibrary{arrows.meta, calc}
\begin{tikzpicture}[line join=round, line cap=round, scale=2]
    % Style for vertices
    \tikzset{vertex/.style={circle, fill=white, draw=black, inner sep=1pt, font=\small}}
    
    % --- DIAGRAM 1: 120-degree rotation axis ---
    \begin{scope}[shift={(0,0)}]
        % Coordinates
        \coordinate (V1) at (0, 1.5, 0);
        \coordinate (V2) at (-0.8, 0, 0.5);
        \coordinate (V3) at (0.8, 0, 0.5);
        \coordinate (V4) at (0, 0, -0.8);

        % Hidden edge
        \draw[dashed, gray] (V2) -- (V4);
        
        % Axis of rotation (Through Vertex 1 and center of opposite face)
        \draw[thick, red, dashed] (0, 1.9, 0) -- (0, -0.5, 0);
        
        % Rotation indicator arc at the top
        \draw[red, -{Stealth}] (0.2, 1.7) arc (0:270:0.2 and 0.08);
        \node[red, right, font=\tiny] at (0.2, 1.75) {$120^\circ$};

        % Solid edges
        \draw[thick] (V1) -- (V2) -- (V3) -- (V1);
        \draw[thick] (V1) -- (V4) -- (V3);

        % Vertices
        \node[vertex] at (V1) {1};
        \node[vertex] at (V2) {2};
        \node[vertex] at (V3) {3};
        \node[vertex] at (V4) {4};

        \node at (0, -0.8) {Rotation $(2\,3\,4)$};
    \end{scope}

    % --- DIAGRAM 2: 180-degree rotation axis ---
    \begin{scope}[shift={(2.5,0)}]
        % Coordinates
        \coordinate (V1) at (0, 1.5, 0);
        \coordinate (V2) at (-0.8, 0, 0.5);
        \coordinate (V3) at (0.8, 0, 0.5);
        \coordinate (V4) at (0, 0, -0.8);
        
        % Midpoints for the axis (edges 1-3 and 2-4)
        \coordinate (M13) at ($(V1)!0.5!(V3)$);
        \coordinate (M24) at ($(V2)!0.5!(V4)$);

        % Hidden edges
        \draw[dashed, gray] (V2) -- (V4);
        \draw[dashed, gray] (V1) -- (V4);
        
        % Axis of rotation (Through midpoints of opposite edges)
        \draw[thick, red, dashed] ($(M13)!-0.6!(M24)$)--($(M24)!-0.6!(M13)$);
        
        % Rotation indicator arc around the axis
        % (Placed near the top midpoint)
        \draw[red, -{Stealth}] ($(M13)+(0.1, 0.2)$ ) arc (0:270:0.15 and 0.06);
        \node[red, right, font=\tiny] at ($(M13)+(0.1, 0.25)$) {$180^\circ$};

        % Solid edges
        \draw[thick] (V1) -- (V2) -- (V3) -- (V1);
        \draw[thick] (V4) -- (V3);

        % Vertices
        \node[vertex] at (V1) {1};
        \node[vertex] at (V2) {2};
        \node[vertex] at (V3) {3};
        \node[vertex] at (V4) {4};

        \node at (0, -0.8) {Rotation $(1\,3)(2\,4)$};
    \end{scope}

\end{tikzpicture}
\end{center}    
    
    
    \begin{itemize}
        \item A rotation of the tetrahedron corresponds to a permutation of its 4 vertices. For example, a $120^\circ$ rotation about a vertex fixies one vertex and cyclically permutes the other three:
        \[ (2\,3\,4) = (2\,4)(2\,3) \in A_4 \]
        \item A $180^\circ$ rotation about the midpoints of opposite edges corresponds to:
        \[ (1\,3)(2\,4) \in A_4 \]
    \end{itemize}

    
    \item If we allow reflections (not just rotations), we get the full symmetric group, e.g. $(2\,4) \in S_4 \setminus A_4$ comes from the reflection:
\begin{center}
\begin{tikzpicture}[line join=round, line cap=round, scale=2]
    % Style for vertices
    \tikzset{vertex/.style={circle, fill=white, draw=black, inner sep=1pt, font=\small}}
    
    % Coordinates (Consistent with previous diagrams)
    \coordinate (V1) at (0, 1.5, 0);
    \coordinate (V2) at (-0.8, 0, 0.5);
    \coordinate (V3) at (0.8, 0, 0.5);
    \coordinate (V4) at (0, 0, -0.8);
    
    % Midpoint of the edge being swapped (2-4)
    \coordinate (M24) at ($(V2)!0.5!(V4)$);

    % --- 1. Draw back/hidden parts ---
    \draw[dashed, gray] (V2) -- (V4);
    \draw[dashed, gray] (V1) -- (V4);

    % --- 2. Draw the Reflection Plane ---
    % The plane passes through V1, V3, and the midpoint of V2-V4
    \fill[red, opacity=0.2] (V1) -- (V3) -- (M24) -- cycle;
    \draw[red, thick, dashed] (V1) -- (M24);
    \draw[red, thick] (V1) -- (V3);
    \draw[red, thick, dashed] (V3) -- (M24);

    % --- 3. Draw the solid edges of the tetrahedron ---
    \draw[thick] (V1) -- (V2) -- (V3) -- cycle;
    \draw[thick] (V3) -- (V4);

    % --- 4. Vertices ---
    \node[vertex] at (V1) {1};
    \node[vertex] at (V2) {2};
    \node[vertex] at (V3) {3};
    \node[vertex] at (V4) {4};

    % Label
    \node at (0, -0.8) {Reflection $(2\,4)$};
    
    % Small text explaining the plane
    \node[red, font=\tiny, align=center] at (1.2, 1.2) {Reflection Plane \\ fixes 1 and 3};

\end{tikzpicture}
\end{center}
    
    
    \item \textbf{Galois Theory}: $A_5$ is a simple group (we will define simple group later). This is a purely group theoretic observation, but it is used by Galois to showing that a general quintic (degree-$5$) equation {\bf cannot} be solved using radicals, i.e. $+, \times, \frac{*}{*}, \sqrt[k]{*}$.
\end{enumerate}
\end{example}
\subsection*{Dihedral Groups $D_n$}
We begin by giving an informal definition of the dihedral group $D_n$: 
\begin{center}
The dihedral group $D_n$ describes the rotational and reflectional symmetries of a regular $n$-gon. Here are the pictures for $n = 3, 4, 5$:
\end{center}

\begin{tikzpicture}[scale=1.2,
    polygon/.style={thick},
    axis/.style={blue,dashed,thick},
    rot/.style={red,thick,-{Stealth[length=3mm]}}
]

% =========================
% Regular 3-gon (Triangle)
% =========================
\begin{scope}[shift={(0,0)}]
    \node at (0,2.3) {\textbf{Regular 3-gon}};
    
    % vertices
    \foreach \i in {0,1,2}
        \coordinate (T\i) at ({90+120*\i}:1.2);
    
    % polygon
    \draw[polygon] (T0)--(T1)--(T2)--cycle;
    
    % reflection axes (FULL lines)
    \foreach \i in {0,1,2}
        \draw[axis]
            ({90+120*\i}:1.8) -- ({90+120*\i+180}:1.8);
    
    % rotation arrow
    \draw[rot] (0.6,0) arc (0:110:0.6);
    \node[red] at (0.9,0.5) {$120^\circ$};
\end{scope}

% =========================
% Regular 4-gon (Square)
% =========================
\begin{scope}[shift={(5,0)}]
    \node at (0,2.3) {\textbf{Regular 4-gon}};
    
    % vertices
    \foreach \i in {0,1,2,3}
        \coordinate (S\i) at ({45+90*\i}:1.2);
    
    % polygon
    \draw[polygon] (S0)--(S1)--(S2)--(S3)--cycle;
    
    % reflection axes
    \draw[axis] (-1.8,0)--(1.8,0);
    \draw[axis] (0,-1.8)--(0,1.8);
    \draw[axis] (-1.6,-1.6)--(1.6,1.6);
    \draw[axis] (-1.6,1.6)--(1.6,-1.6);
    
    % rotation arrow
    \draw[rot] (0.6,0) arc (0:90:0.6);
    \node[red] at (0.9,0.4) {$90^\circ$};
\end{scope}

% =========================
% Regular 5-gon (Pentagon)
% =========================
\begin{scope}[shift={(10,0)}]
    \node at (0,2.3) {\textbf{Regular 5-gon}};
    
    % vertices
    \foreach \i in {0,...,4}
        \coordinate (P\i) at ({90+72*\i}:1.2);
    
    % polygon
    \draw[polygon]
        (P0)--(P1)--(P2)--(P3)--(P4)--cycle;
    
    % reflection axes (FULL lines)
    \foreach \i in {0,...,4}
        \draw[axis]
            ({90+72*\i}:1.8) -- ({90+72*\i+180}:1.8);
    
    % rotation arrow
    \draw[rot] (0.6,0) arc (0:70:0.6);
    \node[red] at (0.95,0.35) {$72^\circ$};
\end{scope}

\end{tikzpicture}

For example, when $n=3$, $D_3$ describes the symmetries of an equilateral triangle:
    \begin{itemize}
        \item Rotations: $e$, $\sigma_{120^\circ}$, $\sigma_{240^\circ}$.
        \item Reflections: 3 axes of symmetry, each fixing one vertex and swapping the other two.
    \end{itemize}
Hence, $D_3$ has 6 elements.
As for $n=4$, $D_4$ describes the symmetries of a square, namely:
    \begin{itemize}
        \item There are 4 rotations $e$, $\sigma_{90^\circ}$, $\sigma_{180^\circ}$, $\sigma_{270^\circ}$
        \item There are also 4 reflections - 2 reflections on along the diagonals, and 2 reflections along the midpoints of the parallel sides.
    \end{itemize}
Hence, $|D_4| = 8$, and more generally, $|D_n| = 2n$ for all $n \ge 3$.

We now give an algebraic representation of $n$-gon symmetries. Label the vertices of the $n$-gon by $1, 2, \dots, n$. Let $r$ be a rotation by $\left(\frac{360}{n}\right)^\circ$ anticlockwise, , and $s$ be a reflection along the axis passing through vertex $1$. For $n = 9$, it looks like:
\begin{center}
\begin{tikzpicture}[
    scale=2,
    polygon/.style={thick},
    axis/.style={blue,dashed,thick},
    rot/.style={red,thick,-{Stealth[length=2mm]}}
]

% Center
\coordinate (O) at (0,0);

% =========================
% Regular 9-gon
% =========================
\foreach \i in {1,...,9}
{
    % vertices (vertex 1 at the top)
    \coordinate (V\i) at ({90+40*(\i-1)}:1);
}

% Draw polygon
\draw[polygon]
    (V1)--(V2)--(V3)--(V4)--(V5)--(V6)--(V7)--(V8)--(V9)--cycle;

% Label vertices
\foreach \i in {1,...,9}
{
    \node at ({90+40*(\i-1)}:1.15) {\small \i};
}

% =========================
% Reflection s
% =========================
% Axis through vertex 1 and center
\draw[axis] (90:1.4) -- (270:1.4);
\node[blue] at (0,1.55) {$s$};

% =========================
% Rotation r
% =========================
\draw[rot] (0.45,0) arc (0:40:0.45);
\node[red] at (0.65,0.25) {$r$};

\end{tikzpicture}
\end{center}
Then $r^n = e$ and $s^2 = e$. Moreover, if we trace how the vertices are moved along under $r$ and $s$, we have 
$$r\ \longleftrightarrow \ (1\ 2\ 3 \dots n),\quad \quad s\ \longleftrightarrow\ (2\ \ (n-1))(3\ \ (n-2)) \cdots$$

Obviously, all $n$ rotations can be expressed as 
$$r^i,\quad 0 \leq i < n.$$
Indeed, the $n$ reflections in $D_n$ can be expressed as 
$$r^i s\quad 0 \leq i < n.$$ 
As an example, for $n = 9$, the picture of $r^2s$ looks like:

\begin{tikzpicture}[
    scale=1.6,
    polygon/.style={thick},
    axis/.style={blue,dashed,thick},
    rot/.style={red,thick,-{Stealth[length=2mm]}},
    map/.style={thick,-{Stealth[length=2mm]}}
]

% radius (slightly shrunk)
\def\r{0.8}

% =========================
% INITIAL 9-GON
% =========================
\begin{scope}[shift={(0,0)}]
    ;
    \coordinate (O) at (0,0);

    \foreach \i in {1,...,9}
        \coordinate (A\i) at ({90+40*(\i-1)}:\r);

    \draw[polygon]
        (A1)--(A2)--(A3)--(A4)--(A5)--(A6)--(A7)--(A8)--(A9)--cycle;

    \foreach \i in {1,...,9}
        \node at ({90+40*(\i-1)}:1.05*\r) {\small \i};

    \draw[axis] (90:1.2*\r) -- (270:1.2*\r);
\end{scope}

% =========================
% ARROW: s
% =========================
\draw[map] (1.6,0) -- (2.3,0) node[midway,above] {$s$};

% =========================
% AFTER REFLECTION s
% =========================
\begin{scope}[shift={(3.9,0)}]
    ;
    \coordinate (O) at (0,0);

    \foreach \i in {1,...,9}
        \coordinate (B\i) at ({90-40*(\i-1)}:\r);

    \draw[polygon]
        (B1)--(B2)--(B3)--(B4)--(B5)--(B6)--(B7)--(B8)--(B9)--cycle;

    \foreach \i in {1,...,9}
        \node at ({90-40*(\i-1)}:1.05*\r) {\small \i};

    \draw[rot] (0.4,0) arc (0:80:0.4);
\end{scope}

% =========================
% ARROW: r^2
% =========================
\draw[map] (5.5,0) -- (6.2,0) node[midway,above] {$r^2$};

% =========================
% AFTER ROTATION r^2
% =========================
\begin{scope}[shift={(7.8,0)}]
  ;
    \coordinate (O) at (0,0);

    \foreach \i in {1,...,9}
        \coordinate (C\i) at ({90-40*(\i-1)+80}:\r);

    \draw[polygon]
        (C1)--(C2)--(C3)--(C4)--(C5)--(C6)--(C7)--(C8)--(C9)--cycle;

    \foreach \i in {1,...,9}
        \node at ({90-40*(\i-1)+80}:1.05*\r) {\small \i};
\end{scope}
\end{tikzpicture}

This has the same effect as doing the reflection:

\begin{center}
\begin{tikzpicture}[
    scale=1.7,
    polygon/.style={thick},
    axis/.style={blue,dashed,thick},
    map/.style={thick,-{Stealth[length=2mm]}}
]

% radius
\def\r{0.8}

% =========================
% ORIGINAL 9-GON
% =========================
\begin{scope}[shift={(0,0)}];
    \coordinate (O) at (0,0);

    % vertices (vertex 1 at top)
    \foreach \i in {1,...,9}
        \coordinate (A\i) at ({90+40*(\i-1)}:\r);

    % polygon
    \draw[polygon]
        (A1)--(A2)--(A3)--(A4)--(A5)--(A6)--(A7)--(A8)--(A9)--cycle;

    % labels
    \foreach \i in {1,...,9}
        \node at ({90+40*(\i-1)}:1.05*\r) {\small \i};

    % reflection axis through vertex 2
    \draw[axis]
        ({90+40}:1.2*\r) -- ({90+40+180}:1.2*\r);
    \node[green] at ({90+40}:1.35*\r){} ;
\end{scope}

% =========================
% ARROW
% =========================
\draw[map] (1.7,0) -- (2.6,0) node[midway,above] {};

% =========================
% REFLECTED 9-GON
% =========================
\begin{scope}[shift={(4.3,0)}];
    \coordinate (O) at (0,0);

    % reflected vertices (mirror across axis through vertex 2)
    \foreach \i in {1,...,9}
        \coordinate (B\i) at ({130-40*(\i-2)}:\r);

    % polygon
    \draw[polygon]
        (B1)--(B2)--(B3)--(B4)--(B5)--(B6)--(B7)--(B8)--(B9)--cycle;

    % labels
    \foreach \i in {1,...,9}
        \node at ({130-40*(\i-2)}:1.05*\r) {\small \i};
\end{scope}
\end{tikzpicture}
\end{center}

In terms of movement of vertices,
$$r^2s \longleftrightarrow (1\ 3)(4\ 9)(5\ 8)(6\ 7).$$
More generally, $r^ks$ represents a reflection along an axis obtained by rotating the ``axis passing through vertex $1$'' by $180 \times \left(\frac{k}{n}\right)^{\circ}$ anticlockwise.

\begin{definition}
The dihedral group $D_n$ is given by the set:
\[ D_n = \{ e, r, r^2, \dots, r^{n-1}, s, rs, r^2s, \dots, r^{n-1}s \} \]
such that $r^n = e, s^2 = e$ and $s r^l = r^{n-l} s$ for all $0 \leq l \leq n$. In terms of {\it generators and relations} (not to be defined precisely in this course), $D_n$ can be written as
$$D_n = \langle s, r \mid r^n = e, s^2 = e, sr^l = r^{n-l}s \text{ for all } l \rangle.$$
\end{definition}

For example, in $D_5$, one can write 
 $$s^3 r^{14} s^5 r^{-3} s^{16} = s r^{10} r^4 s r^5 r^{-3} e = s r^4 s r^2 = s(s r^1) r^2 = s^2 r^3 = r^3.$$

\subsection*{Product Groups (External Direct Product)}

\begin{definition}[Product Group] 
Let $G_1, \dots, G_n$ be groups. The \textbf{product group} $G := \prod_{i=1}^n G_i = G_1 \times \dots \times G_n$ is given by 
$$G := \{ (g_1, \dots, g_n) \mid g_i \in G_i \}$$ 
whose multiplication is given by:
\[ (g_1, \dots, g_n) * (h_1, \dots, h_n) := (g_1 h_1, g_2 h_2, \dots, g_n h_n) \]
In particular, $|G| = |G_1| \times \dots \times |G_n|$ and $e_G := (e_1, \dots, e_n) \in G$, where $e_i = e_{G_i}$ is the identity element of $G_i$.
\end{definition}

\begin{example} The product group $\mathbb{Z}_2 \times \mathbb{Z}_3$ 
is given by 
$$\mathbb{Z}_2 \times \mathbb{Z}_3 = \{ ([a], [b]) \mid a = 0, 1;\ b = 0, 1, 2 \}$$
with $(1, 2) * (0, 1) = (1+0, 2+1) = (1, 3) \equiv (1, 0)$.
(Exercise: $\mathbb{Z}_2 \times \mathbb{Z}_3 = \langle (1, 1) \rangle$ is cyclic.)
\end{example}

One can construct many other groups using products, for instance: 
$$S_3 \times D_4;\quad \mathbb{Z} \times \mathbb{R} ;\quad GL(2, \mathbb{R}) \times S_5 \times \mathbb{Z}_7;\quad \dots$$

\section{Homomorphism and Isomorphism}

\noindent \textbf{Motivation:} Recall that in understanding $D_4$, one can look at how the vertices of the squares are permuted. This gives the following `dictionary' between elements of $D_4$ and $S_4$:
\begin{center}
\begin{tabular}{c c c | c c c}
$D_4$ & & $S_4$ & $D_4$ & & $S_4$ \\
\hline
$e$ & $\leftrightarrow$ & $e$ & $s$ & $\leftrightarrow$ & $(2\,4)$ \\
$r$ & $\leftrightarrow$ & $(1\,2\,3\,4)$ & $rs$ & $\leftrightarrow$ & $(1\,2)(3\,4)$ \\
$r^2$ & $\leftrightarrow$ & $(1\,3)(2\,4)$ & $r^2s$ & $\leftrightarrow$ & $(1\,3)$ \\
$r^3$ & $\leftrightarrow$ & $(1\,4\,3\,2)$ & $r^3s$ & $\leftrightarrow$ & $(1\,4)(2\,3)$ \\
\end{tabular}
\end{center}
To understand $D_4$, it is equally good to understand the 8 elements in $S_4$ in the above dictionary.

\begin{definition}[Homomorphism \& Isomorphism]
Let $(G, \ast), (H, \circledast)$ be groups. A \textit{homomorphism} from $G$ to $H$ is a map $\phi : G \to H$ such that 
$$\phi(g_1 \ast g_2) = \phi(g_1) \circledast \phi(g_2).$$ 
If $\phi$ is bijective, then $\phi$ is an \textit{isomorphism}. 
\end{definition}

\begin{example}
Here is a long list of (non-)examples of group homomorphisms:
\begin{itemize}
    \item $\phi : D_4 \to S_4$ given by $\phi(r) = (1\,2\,3\,4)$, $\phi(s) = (2\,4)$ and so on, as in the above dictionary. Then $\phi$ is an \textit{injective} homomorphism. For instance, 
    $$\phi({\color{red} r^2} {\color{blue} s}) = (1\ 3) = {\color{red} (1\ 3)(2\ 4)} {\color{blue} (2\ 4)} = {\color{red} \phi(r^2)} {\color{blue} \phi(s)}$$
    
    \item $\phi : \mathbb{Z}_3  \to S_4$ given by 
    $$\phi([0])=e,\quad \phi([1]) := (1\ 2 \ 4),\quad \phi([2]) :=(1\ 4\ 2),$$
    then $\phi$ is a homomorphism. For instance, 
    $$\phi([1] + [1]) = \phi([2]) = (1\ 4\ 2) = (1\ 2\ 4) \ast (1\ 2\ 4) = \phi([1]) \ast \phi([1])$$
    
    \item Let $(G, \ast) = (\mathbb{R}^n, +)$ and $(H, \circledast) = (\mathbb{R}^m, +)$. Then any linear transformation 
    $$T : \mathbb{R}^n \to \mathbb{R}^m$$ 
    is a group homomorphism, since
    $$T(x_1 + x_2) = T(x_1) + T(x_2)$$
    
    \item $\phi : (\mathbb{Z}_{15}^*, \times) \to (\mathbb{Z}_{15}, +)$ given by 
    $$\phi([k]) := [k]$$ 
    is \textbf{NOT} a homomorphism, since
    $$\phi([2] \cdot [4]) = \phi([8]) = [8] \neq [6] =[2] + [4] = \phi(2) + \phi(4)$$
    
    \item $\exp : (\mathbb{R}, +) \to (\mathbb{R}_{> 0}, \cdot)$ given by $\exp(a) := e^a$ is a homomorphism (here $\mathbb{R}_{> 0}$ is the set of all positive real numbers), since:
    $$\exp(a+b) = e^{a+b} = e^a \cdot e^b = \phi(a) \cdot \phi(b)$$
    Indeed, it is an isomorphism since $\exp: \mathbb{R} \to \mathbb{R}_{>0}$ is bijective.
    
    \item $S_3$ and $\mathbb{Z}_6$ both have 6 elements, but there are no bijective homomorphism $\phi: S_3 \to \mathbb{Z}_6$ (see below). Similarly, there are not bijective homomorphism from $D_4$ to $\mathbb{Z}_8$.
    
    \item $i : (\mathbb{Z}, +) \hookrightarrow (\mathbb{R}, +)$ defined by $i(a) := a$ for all $a \in \mathbb{Z}$ is a homomorphism. More generally, if $H \le G$, then $i : (H, \ast) \hookrightarrow (G, \ast)$ is a homomorphism.
    
    \item $\pi : (\mathbb{Z}, +) \to (\mathbb{Z}_n, +)$ with 
    $$\pi(a) := [a]_n$$ 
    is a homomorphism, since
    $$\pi(a+b) = [a+b]_n = [a]_n + [b]_n = \pi(a) + \pi(b).$$
    
    \item $\det : GL(n, \mathbb{R}) \to (\mathbb{R}^*, \cdot)$ is a homomorpism, since
    $$\det(A \cdot B) = \det(A) \cdot \det(B).$$
    
    \item Let $\phi : S_n \to (\{ \pm 1 \}, \cdot)$ by $$\phi(\sigma) := \begin{cases} +1, & \text{if } \sigma \text{ is even} \\ -1, & \text{if } \sigma \text{ is odd} \end{cases}$$
    Then $\phi$ is a surjective homomorphism, i.e.
    $$\phi(\sigma \tau) = \phi(\sigma) \cdot \phi(\tau) \quad \forall \sigma, \tau \in S_n,$$ 
    since: 
    \begin{center}
    \begin{tabular}{l l l c c c c}
    & & & $\phi(\sigma \tau)$ & & $\phi(\sigma) \phi(\tau)$ & \\
    \hline
    $\sigma$ even, & $\tau$ even & $\Rightarrow \sigma \tau$ even & $1$ & $=$ & $1 \cdot 1$ & \\
    $\sigma$ even, & $\tau$ odd & $\Rightarrow \sigma \tau$ odd & $-1$ & $=$ & $1 \cdot (-1)$ & \\
    $\sigma$ odd, & $\tau$ even & $\Rightarrow \sigma \tau$ odd & $-1$ & $=$ & $(-1) \cdot 1$ & \\
    $\sigma$ odd, & $\tau$ odd & $\Rightarrow \sigma \tau$ even & $1$ & $=$ & $(-1) \cdot (-1)$ & \\
    \end{tabular}
    \end{center}
\end{itemize}
\end{example}

\begin{proposition} \label{prop:gphomomproperty}
Let $(G, \ast), (H, \circledast)$ and $(K, \star)$ be groups. Then:
\begin{enumerate}
    \item[(a)] If $\phi : G \to H, \psi : H \to K$ are homomorphism, then $\psi \circ \phi : G \to K$ is homomorphism
    \item[(b)] If $\phi : G \to H$ is homomorphism, then $\phi(e_G) = e_H$, $\phi(a^{-1}) = (\phi(a))^{-1}$.
    \item[(c)] If $\phi : G \cong H$ is isomorphism, then $\phi^{-1} : H \to G$ satisfies 
    $$\phi^{-1}(h_1 \circledast h_2) = \phi^{-1}(h_1) \ast \phi^{-1}(h_2).$$ 
    In other words, the inverse of isomorphism is an isomorphism.
\end{enumerate}
\end{proposition}
\begin{proof} 
\begin{itemize}
    \item[(a)] Since $\phi$ and $\psi$ are homomorphisms, $\phi(g_1 * g_2) = \phi(g_1) \circledast \phi(g_2)$ and $\phi(h_1 \circledast h_2) = \phi(h_1) \star \phi(h_2)$. Hence we have:
    \begin{align*}
        (\psi \circ \phi)(g_1 * g_2) = \psi(\phi(g_1 * g_2)) &= \psi(\phi(g_1) \circledast \phi(g_2))\\ 
        &= \psi(\phi(g_1)) \star \psi(\phi(g_2))\\
        &= (\psi \circ \phi)(g_1) \star (\psi \circ \phi)(g_2). 
    \end{align*}

\item[(b)] Since
$\phi(g) = \phi(e_G * g) = \phi(e_G) \circledast \phi(g)$, one has
\begin{align*} \phi(g)(\phi(g))^{-1} &= \phi(e_G) \circledast (\phi(g) \circledast (\phi(g))^{-1}) \\
e_H &= \phi(e_G). \end{align*}
Similarly, since
$\phi(e_G) = \phi(a * a^{-1}) = \phi(a) \circledast \phi(a^{-1})$, therefore 
$$e_H = \phi(a) \circledast \phi(a^{-1}) \Rightarrow \phi(a^{-1}) = (\phi(a))^{-1}$$

\item[(c)] Note that 
$$\phi(\phi^{-1}(h_1 \circledast h_2)) = h_1 \circledast h_2 = (\phi \circ \phi^{-1}(h_1)) \circledast (\phi \circ \phi^{-1}(h_2)) = \phi(\phi^{-1}(h_1) * \phi^{-1}(h_2))$$
Since $\phi$ is bijective, one can apply $\phi^{-1}$ on both sides of the equation and get
$$\phi^{-1}(h_1 \circledast h_2) = \phi^{-1}(h_1) * \phi^{-1}(h_2).$$
\end{itemize}
\end{proof}

\begin{definition}[Kernel and Image]
Let $\phi : G \to H$ be a group homomorphism.
\begin{itemize}
    \item The \textbf{kernel} of $\phi$ is: 
    $$\ker \phi := \{ g \in G \mid \phi(g) = e_H \}.$$
    \item The \textbf{image} of $\phi$ is: 
    $$\text{im } \phi := \phi(G) = \{ \phi(g) \mid g \in G \}.$$
\end{itemize}
\end{definition}
\begin{example}
\begin{itemize}
    \item Let $A$ be an $m \times n$ real matrix, which defines a linear transformation  $A: (\mathbb{R}^n, +) \to (\mathbb{R}^m, +)$. Then $$\ker(A) = \{ \mathbf{v} \in \mathbb{R}^n \mid A(\mathbf{v}) = \mathbf{0}_{\mathbb{R}^m} \} = \text{Null space of } A.$$
    $$\text{im}(A) = \{ A(\mathbf{v}) \in \mathbb{R}^m \mid \mathbf{v} \in \mathbb{R}^n \} = \text{Column space of } A.$$
    
    \item Let $\pi : (\mathbb{Z}, +) \to (\mathbb{Z}_n, +)$. Then 
    $$\ker(\pi) = \{ \text{multiples of } n \} = \langle n \rangle, \quad \text{im}(\pi) = (\mathbb{Z}_n, +).$$
    
    \item Let $\det : GL(n, \mathbb{R}) \to (\mathbb{R}^*, \cdot)$. Then
    $$\ker(\det) = \{ A \in GL(n, \mathbb{R}) \mid \det(A) = 1 \} =: SL(n, \mathbb{R}), \quad \text{im}(\det) = \mathbb{R}^*.$$
\end{itemize}
\end{example}
\begin{proposition}
\begin{itemize}
\item[(a)] $\ker \phi \le G$, $\text{im } \phi \le H$ are subgroups of $G$ and $H$ respectively.
\item[(b)] $\phi$ is an isomorphism if and only if $\ker \phi = \{e_G\}$ and $\text{im } \phi = H$.
\item[(c)] If $G$ is cyclic / abelian, then $\phi(G)$ is cyclic / abelian.
\end{itemize}
\end{proposition}

\begin{proof}
(a) $\forall g_1, g_2 \in \ker \phi$:
\begin{itemize}
    \item[(i)] $\phi(g_1 g_2) = \phi(g_1) \phi(g_2) = e_H \cdot e_H = e_H \Rightarrow g_1 g_2 \in \ker \phi$
    \item[(ii)] $\phi(g_1^{-1}) = (\phi(g_1))^{-1} \Rightarrow g_1^{-1} \in \ker \phi$
\end{itemize}

(b) Skipped. $\left( \ker \phi = \{e_G\} \iff \phi \text{ is injective}; \ \text{im } \phi = H \iff \phi \text{ is surjective} \right)$

(c) If $G$ is abelian, i.e.: $ab = ba \ \forall a, b \in G$. Then $\forall \phi(a), \phi(b) \in \phi(G)$:
    \[ \phi(a) \phi(b) = \phi(ab) = \phi(ba) = \phi(b) \phi(a) \]
    
    
If $G = \langle g \rangle$ is cyclic, then for all $\phi(a) \in \phi(G)$, $a = g^k$ for some $k \in \mathbb{Z}$, and:
    \[ \phi(a) = \phi(g^k) = (\phi(g))^k \in \langle \phi(g) \rangle \]
    (when $k < 0$ is negative,  Proposition \ref{prop:gphomomproperty}(b) is needed to prove the second equality). Therefore, $\phi(G) \subseteq \langle \phi(g) \rangle$. Meanwhile, $\langle \phi(g) \rangle \subseteq \phi(G)$ since $\phi(g) \in \phi(G)$. Therefore, $\phi(G) = \langle \phi(g) \rangle$ is cyclic.
\end{proof}

\begin{example}
\begin{itemize}
    \item $S_3, \mathbb{Z}_6$ both have 6 elements. But $S_3 \not\cong \mathbb{Z}_6$ -- Suppose by contrary that 
    $$\phi : \mathbb{Z}_6 \xrightarrow{\cong} S_3.$$ Then $\phi(\mathbb{Z}_6) = S_3$ and hence $\phi(\mathbb{Z}_6)$ is abelian by the above proposition. However, we already know that $S_3$ is not abelian, so this is impossible.
    
    \item Similarly, $\mathbb{Z}_2 \times \mathbb{Z}_2 \not\cong \mathbb{Z}_4$. (Hint: Suppose on contrary, $\phi : \mathbb{Z}_4 \to \mathbb{Z}_2 \times \mathbb{Z}_2$ and study $\phi(x)$)
\end{itemize}
\end{example}

From now on, we will classify groups up to isomorphism, i.e. we will not distinguish $A_3 \cong \mathbb{Z}_3$ but $\mathbb{Z}_2 \times \mathbb{Z}_2$ and $\mathbb{Z}_4$ are different!


\begin{theorem}[Classification of cyclic groups]
Let $G = \langle g \rangle$ be a cyclic group. Then:
\begin{enumerate}
    \item[(a)] If $|G| = \infty$, then $G \cong (\mathbb{Z}, +)$.
    \item[(b)] If $|G| = n < \infty$, then $G \cong (\mathbb{Z}_n, +)$.
\end{enumerate}
\end{theorem}

\begin{proof}
(a) Consider the map $\phi : \mathbb{Z} \to G = \langle g$ by $\phi(a) := g^a$.
Then 
$$\phi(a+b) = g^{a+b} = g^a g^b = \phi(a) \phi(b).$$ 
Therefore, $\phi$ is a surjective homomorphism. Suppose on contrary that $\phi$ is \textbf{NOT} injective, i.e. there exists $m > n$ such that $\phi(m) = \phi(n)$. Then: 
$$\phi(m) = \phi(n) \ \Rightarrow\ g^m = g^n \Rightarrow\ g^{m-n} = e \Rightarrow\ \text{ord}(g) \le m-n$$ 
(Exercise: $g^k = e\ \Leftrightarrow\ \text{ord}(g) \mid k$). Therefore, $\text{ord}(g) < \infty$ and hence $|G| = |\langle g \rangle| = |\text{ord}(g)| < \infty$ (c.f. Proposition \ref{prop:ordg}), contradicting $|G| = \infty$. Therefore, $\phi$ must be injective, and is an isomorphism between $\mathbb{Z} \cong G$.

\vspace{1em}
\noindent (b) Let $|G| = |\langle g \rangle| = n < \infty$. Then $\text{ord}(g) = n$ by Proposition \ref{prop:ordg} again, with
$$G = \{e =g^0, g^1, \dots, g^{n-1}\}.$$
Consider $\psi : G \to \mathbb{Z}_n$ with $\psi(g^i) := i$ for $0 \leq i \leq n-1$.
Then $\psi$ is a surjective homomorphism, since 
$$\psi(g^a g^b) = \psi(g^{a+b}) = a+b = \psi(g^a) + \psi(g^b).$$ 
Since $|G| = |\mathbb{Z}_n| = n$, then any surjective map is also injective. Therefore, $\psi : G \xrightarrow{\cong} \mathbb{Z}_n$ is bijective.
\end{proof}


\section{Lagrange's Theorem} \label{sec:Lagrange}
\textbf{Recall:} equivalence relation $\sim$ on $S$: If $x \sim y$ in $S$, then $x, y$ are in one equivalence class (with respect to $\sim$). $S = \coprod_{\alpha \in I} C_{\alpha}$

\vspace{0.5em}
\noindent \textbf{e.g.:}
\begin{itemize}
    \item $G = \mathbb{Z}, \ H = 3\mathbb{Z}$ on $\mathbb{Z}$, let $a \sim b \pmod 3$.
    \begin{itemize}
        \item[(i)] $a \sim a \pmod 3 \Rightarrow 3 \mid (a-a)$.
        \item[(ii)] $a \sim b \pmod 3 \Rightarrow b \sim a \pmod 3$ is obvious since $3 \mid (a-b)$.
        \item[(iii)] $a \sim b \pmod 3, \ b \sim c \pmod 3 \Rightarrow a-b = 3k, \ b-c = 3l \Rightarrow a-c = 3(k+l) \in 3\mathbb{Z}$. \\
        $\Rightarrow a \sim c \pmod 3$. $\therefore \sim$ is an equiv. relation.
    \end{itemize}
    $G = \{0+3\mathbb{Z}\} \cup \{1+3\mathbb{Z}\} \cup \{2+3\mathbb{Z}\} = C_0 \cup C_1 \cup C_2$.
    
    \item $G = GL(n, \mathbb{R}), \ H = SL(n, \mathbb{R})$. $A \sim B$ iff $\det(A) = \det(B)$. \\
    (Exercise: check $\sim$ satisfies (i)-(iii)). \\
    $C_{\alpha} = [A] = \{ B \in GL(n, \mathbb{R}) \mid \det(B) = \alpha \}$.
\end{itemize}

\noindent $\bullet$ If $a \sim a'$ in $S$, then $C_a = C_{a'}$. (e.g.: in the example above, $1 \sim 4$). \\
Therefore, any element in an equivalence class $C$ is a representative of $C$.

\noindent $\bullet$ If $a \not\sim a'$, then $C_a \cap C_{a'} = \emptyset$. (e.g.: in the example above, $1 \not\sim 2$, $C_1 \cap C_2 = \emptyset, \ C_0 \cap C_1 = \emptyset \dots$). \\
Therefore, $S$ can be partitioned into disjoint equivalence classes $S = C_1 \cup C_2 \cup \dots$ where $\{C_{\alpha}\}_{\alpha \in I}$ are equivalence classes and $C_{\alpha} \cap C_{\beta} = \emptyset$ if $\alpha \neq \beta$.

\vspace{1em}
\noindent \textbf{e.g.:} $\mathbb{Z} = C_0 \cup C_1 \cup C_2 = (0+3\mathbb{Z}) \cup (1+3\mathbb{Z}) \cup (2+3\mathbb{Z})$ \\
$GL(n, \mathbb{R}) = \coprod_{\alpha \in \mathbb{R}^*} \{ \begin{pmatrix} \alpha & 0 \\ 0 & I \end{pmatrix} SL(n, \mathbb{R}) \} = \coprod_{\alpha \in \mathbb{R}^*} \{ A \mid \det(A) = \alpha \}$.

\vspace{1em}
\begin{definition}[Equivalence Relation on Groups]
Let $G$ be a group and $H \le G$. Define an equivalence relation on $G$ by $a \sim b$ iff $a^{-1}b \in H$.
\end{definition}

\noindent \textbf{(Check):}
\begin{itemize}
    \item[(i)] $a \sim a$, since $a^{-1}a = e \in H$.
    \item[(ii)] $a \sim b \Rightarrow a^{-1}b \in H \Rightarrow (a^{-1}b)^{-1} \in H \Rightarrow b^{-1}a \in H \Rightarrow b \sim a$.
    \item[(iii)] $a \sim b, \ b \sim c \Rightarrow a^{-1}b \in H, \ b^{-1}c \in H \Rightarrow (a^{-1}b)(b^{-1}c) \in H$ \\
    $\Rightarrow a^{-1}c \in H \Rightarrow a \sim c$.
\end{itemize}

\vspace{1em}
\begin{definition}[Left Coset]
Let $G$ be a group and $H \le G$. Define the \textbf{left coset} of $H$ with representative $a$ as $aH := C_a = \{ b \in G \mid a \sim b \} = \{ b \in G \mid a^{-1}b \in H, \text{ for } b \in G \}$ \\
$= \{ b \in G \mid b = ah, \ h \in H \}$.
\end{definition}

\vspace{1em}
\begin{example}
\begin{itemize}
    \item $G = \mathbb{Z}, \ H = 3\mathbb{Z} = \langle 3 \rangle$. The cosets of $3\mathbb{Z}$ in $\mathbb{Z}$ are (under addition): \\
    The left cosets are: $0+3\mathbb{Z} = 3\mathbb{Z}, \ 1+3\mathbb{Z}, \ 2+3\mathbb{Z}$.
    
    \item $G = GL(n, \mathbb{R}), \ H = SL(n, \mathbb{R})$. $a \sim b$ iff $a^{-1}b \in SL(n, \mathbb{R}) \Leftrightarrow \det(a^{-1}b) = 1$ \\
    $\Leftrightarrow \det(a) = \det(b)$. \\
    The left cosets are: $C_{\begin{pmatrix} \alpha & 0 \\ 0 & I \end{pmatrix}} = \begin{pmatrix} \alpha & 0 \\ 0 & I \end{pmatrix} SL(n, \mathbb{R}) = \{ B \mid \det(B) = \alpha \} = \{ B \mid \det(B) = \det(\alpha) \}$.
    
    \item $G = S_3, \ H = \langle (1\,2) \rangle = \{e, (1\,2)\}$. \\
    Left cosets: $C_e = eH = \{e, (1\,2)\}$. \\
    $C_{(1\,2\,3)} = (1\,2\,3)H = \{(1\,2\,3), (1\,2\,3)(1\,2)\} = \{(1\,2\,3), (1\,3)\}$. \\
    $C_{(1\,3\,2)} = (1\,3\,2)H = \{(1\,3\,2), (1\,3\,2)(1\,2)\} = \{(1\,3\,2), (2\,3)\}$.
    
    \item $G = D_4, \ H = \langle s \rangle = \{e, s\}$. \\
    $eH = \{e, s\} = sH$. \\
    $rH = \{r, rs\} = rsH$. \\
    $r^2H = \{r^2, r^2s\} = r^2sH$. \\
    $r^3H = \{r^3, r^3s\} = r^3sH$.
\end{itemize}
\end{example}

\begin{remark} 
We can also define $a \sim_R b$ by $ba^{-1} \in H$. Then the equivalence class of $a$ is called \textbf{right coset} $Ha = \{ ha \mid h \in H \}$.
\end{remark}
\vspace{1em}
\begin{theorem}[Lagrange's Theorem]
Let $G$ be a finite group and $H \le G$. Then $|H| \mid |G|$. \\
(e.g.: all subgroups of $G = S_3$ have order 1, 2, 3, 6 only). \\
More precisely, let $m = [G : H] :=$ the number of disjoint left cosets of $H$. \\
Then $|H| = |G| / [G : H]$. \\
e.g.: $\bullet \ G = S_3, \ H = \langle (1\,2) \rangle$. Then $m = 3 \Rightarrow |H| = |S_3| / 3 = 6 / 3 = 2$. \\
\phantom{e.g.:} $\bullet \ G = D_n, \ H = \langle r \rangle$. Then $m = 2 \Rightarrow |H| = |D_n| / 2 = 2n / 2 = n$.
\end{theorem}

\begin{proof}
Since (left cosets of $H$) = (equivalence classes of $G$), we can partition $G = eH \amalg a_2 H \amalg \dots \amalg a_m H$ (let $a_1 := e$) into disjoint union of equivalence classes. \\
($m = [G : H] < \infty$, since $|G| < \infty$). \\
So we just need to check each $a_i H$ have the same number of elements $|H| = r$ \hfill (*) \\
$a_i H = \{ a_i h_1, a_i h_2, \dots, a_i h_r \}$ where $\{ h_1, \dots, h_r \} = H$. Then for each $a_i h_x, a_i h_y \in a_i H$: \\
$a_i h_x = a_i h_y \Leftrightarrow a_i^{-1} a_i h_x = a_i^{-1} a_i h_y \Leftrightarrow h_x = h_y \Leftrightarrow x = y$. \\
So the elements in $a_i H$ are distinct, and hence, $|a_i H| = |H| = r$. \\
$\therefore$ By (*), $|G| = |H| + |H| + \dots + |H| = m|H|$. \hfill $\square$
\end{proof}

\vspace{1em}
\begin{corollary}
Let $|G| < \infty$. Then for all $g \in G, \ \text{ord}(g) \mid |G|$. \\
e.g.: all elements in $G = D_5$ must have order 1, 2, 5, 10 only.
\end{corollary}

\begin{proof}
Take $H := \langle g \rangle \le G$. Then $\text{ord}(g) = |\langle g \rangle| = | \{ e, g, \dots, g^{n-1} \} |$. \\
Then by Lagrange's Thm, $\text{ord}(g) = |\langle g \rangle| \mid |G|$. \hfill $\square$
\end{proof}

\begin{example}[Fermat's Little Theorem] 
Let $a \in G = \mathbb{Z}_p^*$. Then $a^{p-1} = e = 1$ in $\mathbb{Z}_p^*$. \\
(e.g.: In $\mathbb{Z}_7$, $2^6 \equiv 3^6 \equiv \dots \equiv 6^6 \equiv 1 \pmod 7$)
\end{example}
\noindent (Proof): By corollary 2.55, for all $a \in \mathbb{Z}_p^* = G$, $d = \text{ord}(a) \mid p-1 = |\mathbb{Z}_p^*|$. \\
Hence, $a^d = 1$ and $d \cdot \alpha = p-1$ for some integer $\alpha$. Then \\
$a^{p-1} = a^{d\alpha} = (a^d)^\alpha = 1^\alpha = 1$.

\vspace{0.5em}
\noindent More generally, $G = \mathbb{Z}_n^*, \ |G| = \phi(n) = \# \{ r \mid 1 \le r \le n, \ \gcd(r, n) = 1 \}$ \\
($\phi$ Euler's totient function). Then $a^{\phi(n)} \equiv 1 \pmod n$.

\section{Normal Subgroups}

\noindent \textbf{Motivations:} We have $G = (g_1 H) \amalg (g_2 H) \amalg \dots$ (left cosets). Also, \\
$G = (H b_1) \amalg (H b_2) \amalg \dots$ (right cosets).

\vspace{0.5em}
\noindent \textbf{Question:} Do we have $\{ gh \mid h \in H \} = gH = Hg = \{ hg \mid h \in H \}$ in general?

\vspace{0.5em}
\noindent \textbf{Answer:} No. Let's try $G = S_3, \ H = \langle (1\,2) \rangle$. Then \\
$(1\,3)H = \{ (1\,3), (1\,3)(1\,2) \} = \{ (1\,3), (1\,2\,3) \}$ \\
$H(1\,3) = \{ (1\,3), (1\,2)(1\,3) \} = \{ (1\,3), (1\,3\,2) \} \neq (1\,3)H$.

\vspace{1em}
\noindent \textbf{Question:} What kind of $H \le G$ gives $gH = Hg$?

\begin{definition}[Normal Subgroup]
$H \le G$ is a \textbf{normal subgroup} if $gH = Hg \quad \forall g \in G$. (And we'll write $H \triangleleft G$ in such cases).
\end{definition}

\vspace{1em}
\begin{example}
\begin{itemize}
    \item $G = S_n, \ H = A_n$. Then $H \triangleleft G$.
    \item If $G$ is abelian, then any subgroup $H \le G$ is normal. \\
    Proof: $\forall g \in G, \forall h \in H$, $gh = hg$ (since $G$ is abelian). So $gH = Hg$ for all $g$.
    
    \item Similarly, $SL(n, \mathbb{R}) \triangleleft GL(n, \mathbb{R})$ is normal. \\
    Proof: Let $A \in GL(n, \mathbb{R})$ and $H = SL(n, \mathbb{R})$. For any $h \in H$, 
    \[ \det(A h A^{-1}) = \det(A) \det(h) \det(A)^{-1} = 1 \cdot \det(A) \cdot \frac{1}{\det(A)} = 1 \]
    $\Rightarrow A h A^{-1} \in SL(n, \mathbb{R})$, so $AH = HA$.
    
    \item If $H \le G$ and $[G:H] = 2$, then $H \triangleleft G$. \\
    Proof: In this case, $G = H \amalg (gH)$ and $G = H \amalg (Hg)$. \\
    Since $H \cap gH = \emptyset$ and $H \cap Hg = \emptyset$, it must be $gH = Hg$.
    
    \item $\{e\} \triangleleft G$ and $G \triangleleft G$ are always normal subgroups.
\end{itemize}
\end{example}


\vspace{1em}
\begin{theorem} 
Let $H \le G$ be a subgroup. The followings are equivalent: \\
(i) $H \triangleleft G$; \quad (ii) $\forall h \in H, g \in G, \ g h g^{-1} \in H$; \quad (iii) $g H g^{-1} = H \quad \forall g \in G$.
\end{theorem}
\noindent (Proof): \\
((i) $\Rightarrow$ (ii)): By assumption, $gH = Hg \quad \forall g \in G$. $\Rightarrow \forall h \in H, \ g h \in gH = Hg$ \\
$\Rightarrow g h = h' g$ for some $h' \in H$. $\Rightarrow g h g^{-1} = h' \in H$.

\vspace{0.5em}
\noindent ((ii) $\Rightarrow$ (iii)): By (ii), $g H g^{-1} = \{ g h g^{-1} \mid h \in H \} \subseteq H$. $\bullet$ $g^{-1} g H g^{-1} g \subseteq g^{-1} H g \Rightarrow H \subseteq g^{-1} H g$. \\
for all $g \in G$. Then $H \subseteq (g^{-1}) H (g^{-1})^{-1}$ (take ``$g = g^{-1}$''). $\Rightarrow H \subseteq g H g^{-1}$. \\
Therefore, $H = g H g^{-1}$.

\vspace{0.5em}
\noindent ((iii) $\Rightarrow$ (i)): $g H g^{-1} g = H g \quad \forall g \in G \Rightarrow g H = H g \quad \forall g \in G \Rightarrow H \triangleleft G$. \hfill $\square$

\vspace{1.5em}
\begin{corollary} 
Let $\phi : G \to H$ be a homomorphism. Then $\ker \phi \triangleleft G$.

\vspace{0.5em}
\noindent (Proof): By (ii) in theorem 2.59, we only need to show $\forall k \in \ker \phi, \ g \in G, \ g k g^{-1} \in \ker \phi$. \\
Indeed, $\phi(g k g^{-1}) = \phi(g) \phi(k) \phi(g^{-1}) = \phi(g) e_H (\phi(g))^{-1} = \phi(g) (\phi(g))^{-1} = e_H$. \hfill $\square$
\end{corollary}

\vspace{1em}
\begin{example}
\begin{itemize}
    \item $\phi : GL(n, \mathbb{R}) \to \mathbb{R}^* \quad \text{with} \quad \phi(A) = \det(A)$. Then $\ker \phi = \{ A \mid \det(A) = 1 \} = SL(n, \mathbb{R})$. \\
    $\therefore SL(n, \mathbb{R}) \triangleleft GL(n, \mathbb{R})$.
    
    \item $\psi : S_n \to (\{ \pm 1 \}, \times) \quad \text{with} \quad \psi(\sigma) := \text{parity of } \sigma$. Then $\ker \psi = A_n \triangleleft S_n$.
\end{itemize}
\end{example}


\section{Quotient Groups}
\noindent \textbf{Motivation:} Let $H \triangleleft G$. We want to define a group structure on the set of left cosets $G/H := \{ gH \mid g \in G \}$.

\begin{definition}[Quotient Group]
Let $G$ be a group and $H \triangleleft G$. Define the \textbf{quotient group} (or factor group) $G/H$ as the set of left cosets $\{ gH \mid g \in G \}$ with the multiplication:
\[ (aH) * (bH) := (ab)H \]
\end{definition}


\noindent \textbf{(WARNING):} \\
There are lots of repetitions in this set, i.e.: there can be $a_1 \neq a_2$, s.t.: $a_1H=a_2H$

\vspace{0.5em}
\noindent \textbf{(Check):}
\begin{itemize}
    \item $*$ is well-defined: Suppose $aH = a'H$ \& $bH = b'H$ --- (*). \\
    By (*), $h_a := a^{-1} a' \in H$, $h_b := b^{-1} b' \in H$. Then $a'b' = (ah_a)(bh_b) = abh_a^h h_b$ \\
    $\Rightarrow a'b' \in (ab)H \Rightarrow (a'b')H = (ab)H$
\end{itemize}


\begin{itemize}
    \item $(G/H, *)$ is a group:
    \begin{enumerate}
        \item $((aH) * (bH)) * (cH) = (ab)H * (cH) = (abc)H = (aH) * (bc)H$ \\
        \phantom{((aH) * (bH)) * (cH)} $= (aH) * ((bH) * (cH))$
        
        \item $(aH) * (eH) = aH = (eH) * (aH)$, i.e.: $e_{G/H} = eH$.
        
        \item $(aH) * (a^{-1}H) = eH = e_{G/H} = (a^{-1}H) * (aH)$, i.e.: $(aH)^{-1} = a^{-1}H$.
    \end{enumerate}
\end{itemize}





\begin{example}
\begin{itemize}
    \item $G = \mathbb{Z} \ , \ H = \langle n \rangle = n\mathbb{Z} \triangleleft G \ . \ (\text{Exercise : All subgroups of abelian}$ \\
    $\text{group } G \text{ are normal})$ \\
    $G/H = \{ 0 + n\mathbb{Z} \ , \ 1 + n\mathbb{Z} \ , \ 2 + n\mathbb{Z} \ , \ \dots \ , \ (n-1) + n\mathbb{Z} \}$ \\
    For $a + n\mathbb{Z} \ , \ b + n\mathbb{Z} \in G/H \ , \ (a + n\mathbb{Z}) * (b + n\mathbb{Z}) := (a+b) + n\mathbb{Z} = c + n\mathbb{Z}$ \\
    where $c \equiv a+b \pmod n \ \& \ 0 \le c < n$ \\
    $\therefore$ There's an isomorphism $\phi : G/H = \mathbb{Z}/n\mathbb{Z} \xrightarrow{\cong} \mathbb{Z}_n$ with $\phi(a + n\mathbb{Z}) := a \pmod n$ \\
    Remark: From now on, I may interchange $\mathbb{Z}/n\mathbb{Z}$ and $\mathbb{Z}_n$.
    
    \item $A_n \triangleleft S_n \ , \ \text{then } S_n/A_n = \{ A_n \ , \ \tau A_n \} \quad (\tau \text{ is any transposition } \tau = (a \ b))$. \\
    (Recall $|S_n| = n! \ , \ |A_n| = \frac{n!}{2} \ , \ \therefore |S_n/A_n| = 2$) \\
    $\therefore S_n/A_n \cong \mathbb{Z}_2 \ (\cong \mathbb{Z}/2\mathbb{Z})$
    
    \item $SL(n, \mathbb{R}) \trianglelefteq GL(n, \mathbb{R})$.
    \begin{align*}
    GL(n, \mathbb{R})/SL(n, \mathbb{R}) &= \left\{ \begin{pmatrix} x & \dots & 0 \\ \vdots & 1 & \dots \\ 0 & \dots & 1 \end{pmatrix} SL(n, \mathbb{R}) \mid x \in \mathbb{R}^* \right\} \\
    &= \{ A \in GL(n, \mathbb{R}) \mid \det(A) = x \}
    \end{align*}
    \begin{align*}
    & \left( \begin{pmatrix} x & \dots & 0 \\ \vdots & 1 & \dots \\ 0 & \dots & 1 \end{pmatrix} SL(n, \mathbb{R}) \right) 
    \left( \begin{pmatrix} y & \dots & 0 \\ \vdots & 1 & \dots \\ 0 & \dots & 1 \end{pmatrix} SL(n, \mathbb{R}) \right) \\
    &= \left( \begin{pmatrix} x & \dots & 0 \\ \vdots & 1 & \dots \\ 0 & \dots & 1 \end{pmatrix} \begin{pmatrix} y & \dots & 0 \\ \vdots & 1 & \dots \\ 0 & \dots & 1 \end{pmatrix} \right) SL(n, \mathbb{R}) \\
    &= \begin{pmatrix} xy & \dots & 0 \\ \vdots & 1 & \dots \\ 0 & \dots & 1 \end{pmatrix} SL(n, \mathbb{R})
    \end{align*}
    $\therefore GL(nd, \mathbb{R})/SL(n, \mathbb{R}) \cong (\mathbb{R}^*, \times)$
    
    \item $K = \{ e, (12)(34), (13)(24), (14)(23) \} \triangleleft S_4$ \\
    Then $|S_4/K| = |S_4|/|K| = 24/4 = 6 \quad$ Indeed, $S_4/K \cong S_3$
    \end{itemize}
    \end{example}

\section{First Isomorphism Theorem}

\begin{theorem}[First Isomorphism Theorem]
Let $\phi : G \to H$ be a group homomorphism. Then $G/\ker \phi \cong \text{im } \phi$.
\end{theorem}
\vspace{1em}
\begin{example}
\begin{itemize}
    \item $\phi : \mathbb{Z} \to \mathbb{Z}_n$ with $\phi(a) := a \pmod n$ is a homomorphism, with \\
    $\ker \phi = n\mathbb{Z} \ , \ \text{im} \phi = \mathbb{Z}_n$. Then by 1st isomorphism, $\mathbb{Z}/n\mathbb{Z} \cong \mathbb{Z}_n$.
    
    \item $\det : GL(n, \mathbb{R}) \to (\mathbb{R}^*, \times)$ with $\ker(\det) = \{ A \mid \det(A) = 1 \} = SL(n, \mathbb{R})$, \\
    $\text{im}(\det) = \mathbb{R}^*$. Hence, $GL(n, \mathbb{R})/SL(n, \mathbb{R}) \cong \mathbb{R}^*$.
    
    \item $\phi : \mathbb{R} \to GL(2, \mathbb{R})$ with $\phi(x) := \begin{pmatrix} \cos x & -\sin x \\ \sin x & \cos x \end{pmatrix}$. Then \\
    $\ker \phi = 2\pi\mathbb{Z} \ , \ \text{and} \ \text{im} \phi = \left\{ \left. \begin{pmatrix} \cos x & -\sin x \\ \sin x & \cos x \end{pmatrix} \right| 0 \le x < 2\pi \right\} = SO(2)$. \\
    $\therefore \mathbb{R}/2\pi\mathbb{Z} \cong SO(2)$.
\end{itemize}
\end{example}
\vspace{1em}

\noindent \textbf{Proof of 1st isomorphism theorem:} Define $\Psi : G/\ker \phi \to \text{im} \phi$ by \\
$\Psi(g \ker \phi) := \phi(g)$.

\begin{enumerate}
    \item[(1)] $\Psi$ is well-defined: For $g \ker \phi = g' \ker \phi \ , \ g^{-1}g' \in \ker \phi \Leftrightarrow e_H = \phi(g^{-1}g')$ \\
    $= \phi(g)^{-1} \phi(g') \Leftrightarrow \phi(g) e_H = \phi(g) \phi(g)^{-1} \phi(g') = \phi(g') \Leftrightarrow \phi(g) = \phi(g')$. \\
    $\Leftrightarrow \Psi(g \ker \phi) = \Psi(g' \ker \phi)$.
    
    \item[(2)] $\Psi$ is a homomorphism: $\Psi((g \ker \phi)(g' \ker \phi)) = \Psi(gg' \ker \phi) := \phi(gg') = \phi(g)\phi(g')$ \\
    $= \Psi(g \ker \phi) \Psi(g' \ker \phi)$.
    
    \item[(3)] $\Psi$ is surjective: Obvious from definition of $\Psi$.
    
    \item[(4)] $\Psi$ is injective: $\forall \ g \ker \phi \in \ker \Psi \ , \ \phi(g) = e_H \Rightarrow g \in \ker \phi$ \\
    $\Rightarrow g \ker \phi = e_G \ker \phi = e_{G/\ker \phi} \Rightarrow \ker \Psi = \{ e_{G/\ker \phi} \}$. \hfill $\square$
\end{enumerate}

\vspace{1em}

\subsection*{Applications of quotient groups}
\begin{itemize}
    \item We can construct new groups from old ones.
    \item Cauchy's Theorem: Suppose $|G| < \infty$ is finite and $p \mid |G|$ for some prime $p$. Then $G$ must have an element $g$ of order $p$.
    \item Classification of finite groups.
\end{itemize}

\section{Simple Groups}

\begin{definition}[Simple group]
A group $G$ is called \textbf{simple} if $G$ has no normal subgroups other than $\{e\}$ and $G$ itself.
\end{definition}
\vspace{0.5em}
\noindent \textbf{Reason for this definition:} \\
If $G$ is \textbf{NOT} simple, then we have $N \triangleleft G$. So we can ``decompose'' $G$ into $G/N$ and $N$, and study them individually.

\vspace{0.5em}
\noindent \textbf{Remark/WARNING:} If $|G| < \infty$ is NOT simple, then $|G/N| = |G| / |N|$. \\
So $|G/N \times N| = |G/N| \times |N| = |G|/|N| \times |N| = |G|$. But $G \not\cong G/N \times N$ in general.

\vspace{1.5em}
\begin{example}
\begin{itemize}
    \item $S_n$ is \textbf{NOT} simple, since $A_n \triangleleft S_n$. So we can ``understand'' $S_n$ by understanding $A_n$ and $S_n/A_n \cong \mathbb{Z}_2$ individually. \\
    But $S_n \not\cong A_n \times (S_n/A_n)$.
    
    \item $A_4$ is \textbf{NOT} simple, since $K = \{ e, (12)(34), (13)(24), (14)(23) \} \triangleleft A_4$. \\
    However, $A_n$ is simple for $n \ge 5$ (very important in Galois theory: there're no radical solutions for quintic polynomial $x^5 + a_4x^4 + a_3x^3 + \dots + a_0 = 0$).
\end{itemize}
\end{example}

\section{Fundamental Theorem of Finite Abelian Groups}

\noindent \textbf{Observations:}
\begin{enumerate}
    \item[(1)] $\mathbb{Z}_n \ (\cong \mathbb{Z}/n\mathbb{Z} = \mathbb{Z}/\langle n \rangle)$ is abelian
    \item[(2)] $G$ and $H$ are abelian $\iff G \times H$ is abelian
    \item[(3)] If $\gcd(m, n) = 1$, then $\mathbb{Z}_{mn} \cong \mathbb{Z}_m \times \mathbb{Z}_n$
\end{enumerate}

\vspace{1em}
\begin{theorem}
All finite abelian groups are of the form $G \cong \mathbb{Z}_{p_1^{a_1}} \times \dots \times \mathbb{Z}_{p_k^{a_k}}$ where $p_1 \le \dots \le p_k$ are primes, $a_1, \dots, a_k \in \mathbb{N} \setminus \{0\}$ ($p_i$ NOT necessarily distinct).
\end{theorem}
\vspace{1em}
\noindent \textbf{(Proof):} There are 2 main steps to prove theorem 2.68: \\
\textbf{Step 1:} If $|G| = q_1^{a_1} \dots q_r^{a_r}$, $q_i$ distinct primes, then $G \cong G_1 \times \dots \times G_r, |G_i| = q_i^{a_i}$ \\
\textbf{Step 2:} If $|G| = q^b$, then $G \cong \mathbb{Z}_{q^{b_1}} \times \dots \times \mathbb{Z}_{q^{b_r}}, \sum_{j=1}^r b_j = b$

\vspace{1em}
\noindent \textbf{Proof of step 1:} Suppose $|G| = p^a q^b$ ($p, q$ distinct primes). We want to show $G \cong G_1 \times G_2$, $|G_1| = p^a =: m$, $|G_2| = q^b =: n$, and the general case follows from induction on \# of prime powers.

\vspace{0.5em}
\noindent Let $G^m := \{ g^m \mid g \in G \} \le G$ \quad \& \quad $G^n := \{ g^n \mid g \in G \} \le G$.

\vspace{0.5em}
\noindent \underline{\textbf{Claim 1:}} $G^m \cap G^n = \{e\}$. Indeed, suppose $x \in G^m \cap G^n$. Then $x = g^m = h^n$ for some $g, h \in G$. Then $x^n = g^{mn} = e$ \& $x^m = h^{nm} = e$ (by Lagrange's Theorem, $|G| = mn$). Note that $\gcd(m, n) = \gcd(p^a, q^b) = 1$ \\
$\Rightarrow \alpha m + \beta n = \gcd(m, n) = 1$ for some $\alpha, \beta \in \mathbb{Z}$. $\Rightarrow x^{\alpha m + \beta n} = (x^m)^\alpha (x^n)^\beta = x^1$ \\
$\Rightarrow e \cdot e = x^1 \Rightarrow x = e$. Then \underline{claim 1 holds}.

\vspace{0.5em}
\noindent \underline{\textbf{Claim 2:}} $\phi: G^m \times G^n \to G$ with $\phi(g^m, h^n) := g^m(h^n)^{-1}$ is an isomorphism:
\begin{itemize}
    \item $\phi$ is a homomorphism. (exercise, need $G$ is abelian)
    \item $\phi$ is injective: Suppose $\phi(g^m, h^n) = e$. Then $g^m(h^n)^{-1} = e \Rightarrow g^m = h^n$ \\
    $\Rightarrow g^m, h^n \in G^m \cap G^n = \{e\} \Rightarrow g^m = h^n = e \Rightarrow \ker \phi = \{(e, e)\}$.
    \item $\phi$ is surjective: Take $y \in G$. \\
    $y = y^1 = y^{\alpha m + \beta n} = (y^\alpha)^m (y^\beta)^n = \phi((y^\alpha)^m, (y^{-\beta})^n)$ \\
    Hence, $\phi$ is an isomorphism.
\end{itemize}

\vspace{0.5em}
\noindent \underline{\textbf{Claim 3:}} $|G^m| = n = q^b$, $|G^n| = m = p^a$ \\
Begin by showing $\gcd(|G^n|, q) = 1$. Suppose $q \mid |G^n|$. Then by Cauchy's Theorem, there exists $x = g^n$ in $G^n$ of order $q$. \\
By Bezout, $\gcd(q, m) = 1$. So one has $\alpha q + \beta m = 1$, and $x^q = e$. Then $x^{\alpha q} = e \Rightarrow x^{1-\beta m} = e \Rightarrow x \cdot (x^m)^{-\beta} = e \Rightarrow x(g^{nm})^{-\beta} = e$ \\
$\xrightarrow{\text{Lagrange's}} x \cdot e^{-1} = e \Rightarrow x = e$, contradicting order of $x$ is $q$. \\
So $\gcd(|G^n|, q) = 1 \Rightarrow |G^n| = p^r$ \& $\gcd(|G^m|, p) = 1 \Rightarrow |G^m| = q^s$ \\
But $G^m \times G^n \cong G$, so $p^r \cdot q^s = |G| = p^a q^b$. Then $a=r, b=s$. \\
Therefore, \underline{claim 3 holds}.

\vspace{1em}
\noindent \textbf{Proof of step 2:} Suppose $|G| = p^a$. Then for all $x \in G$, $\text{ord}(x) = p^i$ for some $0 \le i \le a$ by Lagrange's Theorem. \\
Let $m \in G$ be an element with maximum order $\text{ord}(m) = p^l$ in $G$.

\vspace{0.5em}
\noindent \underline{\textbf{Proposition:}} Let $G$ be an abelian group with $|G| = p^a$. Suppose $\mu \in G$ has maximum order $\text{ord}(\mu) = p^l$ in $G$, then $G \cong \langle \mu \rangle \times K \cong \mathbb{Z}_{p^l} \times K$.

\vspace{0.5em}
\noindent \textbf{Proof of Proposition:} Induction on $p^a = |G|$. If $a=1$, then $|G| = p \Rightarrow G \cong \mathbb{Z}_p$. \\
Then suppose the Proposition holds for all abelian $G$ with $|G| = p^r, r < a$. \\
Now let $|G| = p^a$. Suppose $\text{ord}(\mu) = p^m$ is maximal:
\begin{itemize}
    \item If $m=a$, then $\langle \mu \rangle = G$ and then we're done.
    \item If $m < a$, then let $\nu \in G$ be such that $\nu$ is non-identity element NOT in $\langle \mu \rangle$ with smallest possible order $p^k$.
\end{itemize}

\vspace{0.5em}
\noindent \underline{\textbf{Claim 4:}} $\text{ord}(\nu) = p$. $\because \text{ord}(\nu^p) < p^k$, so $\nu^p \in \langle \mu \rangle$ and $\nu^p = \mu^i$ for some $i \in \mathbb{Z}$. Then $(\mu^i)^{p^{m-1}} = (\nu^p)^{p^{m-1}} = \nu^{p^m} = e$, since every $g \in G$ has order $\le p^m$. \\
$\Rightarrow \text{ord}(\mu^i) \mid p^{m-1} \Rightarrow |\langle \mu^i \rangle| < p^m \Rightarrow \langle \mu^i \rangle \neq \langle \mu \rangle \Rightarrow \gcd(i, p^m) > 1$ \\
$\therefore i = pj$ for some $j \in \mathbb{Z}$. Hence, $\nu^p = \mu^{pj}$ for some $j \in \mathbb{Z}$. \\
Let $c := \nu \mu^{-j}$. Then $c \notin \langle \mu \rangle$ and $c^p = e$, since $\nu \notin \langle \mu \rangle$ is chosen to have smallest order. Therefore, $\text{ord}(\nu) = p$. $\Rightarrow$ \underline{claim 4 holds}.

\vspace{0.5em}
\noindent \underline{\textbf{Claim 5:}} $\langle \mu \rangle \cap \langle \nu \rangle = \{e\}$. Let $\nu^l \in \langle \mu \rangle \cap \langle \nu \rangle$. Then $0 \le l < p$, since $\text{ord}(\nu) = p$. Suppose $l \neq 0$, then $\gcd(l, p) = 1 \Rightarrow \alpha l + \beta p = 1$ \\
$\Rightarrow \langle \mu \rangle \cap \langle \nu \rangle \ni (\nu^l)^\alpha = \nu \cdot \nu^{-\beta p} = \nu \Rightarrow \nu \in \langle \mu \rangle$. Contradiction. \\
Therefore, $\langle \mu \rangle \cap \langle \nu \rangle = \{e\}$, i.e.: \underline{claim 5 holds}.

\vspace{0.5em}
\noindent Let $\bar{G} := G/\langle \nu \rangle$, s.t.: $|\bar{G}| = \frac{|G|}{|\langle \nu \rangle|} = \frac{p^a}{p} = p^{a-1}$ \& write $\bar{g} := g\langle \nu \rangle \in \bar{G}$.

\vspace{0.5em}
\noindent \underline{\textbf{Claim 6:}} $\bar{\mu} \in \bar{G}$ has order $p^m$ (maximal order). Obviously, $\bar{\mu}^{p^m} = (\mu\langle \nu \rangle)^{p^m} = \mu^{p^m}\langle \nu \rangle = e\langle \nu \rangle = e_{\bar{G}} \Rightarrow \text{ord}(\bar{\mu}) \mid p^m$ \\
Suppose by contrary, $\text{ord}(\bar{\mu}) = p^u$ for $u < m$. Then $\mu^{p^u}\langle \nu \rangle = e_{\bar{G}} = e\langle \nu \rangle$ \\
$\Rightarrow \mu^{p^u} \in \langle \nu \rangle \Rightarrow \mu^{p^u} \in \langle \mu \rangle \cap \langle \nu \rangle \xrightarrow{\text{claim 5}} \{e\} \Rightarrow \mu^{p^u} = e$, contradicting $\text{ord}(\mu) = p^m$. Therefore, \underline{claim 6 holds}.

\vspace{0.5em}
\noindent By induction, there is another subgroup $\bar{K} \le \bar{G}$ s.t.: $\langle \bar{\mu} \rangle \times \bar{K} \cong \bar{G} \ (*)$. \\
That is, every element in $\bar{G}$ can be written \underline{uniquely} as $\bar{\mu}^i \bar{k} \in \bar{G}$ for some $\bar{\mu}^i \in \langle \bar{\mu} \rangle, \bar{k} \in \bar{K}$. Let $\pi: G \to \bar{G}$ be given by $\pi(g) := \bar{g}$. Then $K := \pi^{-1}(\bar{K}) = \{ k \in G \mid \pi(k) \in \bar{K} \} \le G$.

\vspace{0.5em}
\noindent \underline{\textbf{Claim 7:}} $\pi|_K: K \to \bar{K}$ has $\ker(\pi|_K) = \langle \nu \rangle$. $x \in \ker(\pi|_K) \iff \pi(x) = x\langle \nu \rangle = e_{\bar{G}} = \langle \nu \rangle \iff x \in \langle \nu \rangle$. Therefore, \underline{claim 7 holds}.

\vspace{0.5em}
\noindent Hence, by 1st isomorphism theorem, $|K/\langle \nu \rangle| = |\bar{K}|. \Rightarrow |K|/p \overset{(*)}{=} \frac{|G|}{|\langle \mu \rangle|} \Rightarrow |K| = \frac{p \cdot p^{a-1}}{p^m} \Rightarrow |K| = p^{a-m}$

\vspace{0.5em}
\noindent \underline{\textbf{Claim 8:}} $\langle \mu \rangle \cap K = \{e\}$ in $G$. Let $\mu^i \in \langle \mu \rangle \cap K$, then $\pi(\mu^i) = \bar{\mu}^i \cdot e_{\bar{K}} = e_{\langle \bar{\mu} \rangle} \cdot \bar{k}$. By uniqueness, $\bar{\mu}^i = e_{\langle \bar{\mu} \rangle}$ \& $\bar{k} = e_{\bar{K}}$ \\
$\Rightarrow \mu^i \in \langle \nu \rangle \Rightarrow \mu^i \in \langle \mu \rangle \cap \langle \nu \rangle = e$. Then \underline{claim 8 holds}.

\vspace{0.5em}
\noindent Finally, consider the homomorphism $\theta: \langle \mu \rangle \times K \to G$ with $\theta(\mu^i, k) := \mu^i k$ \\
Then by claim 8, $\theta$ is injective. By claim 7, $|\langle \mu \rangle \times K| = |\langle \mu \rangle| |K| = p^m \cdot p^{a-m} = p^a = |G|$. Hence, $\theta$ is surjective as well, and hence: \\
$\langle \mu \rangle \times K \cong G$ \hfill $\square$


\vspace{1.5em}
\begin{example}
$|G|=4$. Then all elements of $G$ have order $1, 2, 4$.
\begin{enumerate}
    \item[(i)] If $G$ has an element $m$ of order $4$, then $G = \langle m \rangle \cong \mathbb{Z}_4$.
    \item[(ii)] Suppose all elements of $G$ has order $1$ or $2$. Take any $m \in G$ of order $2$, $G \cong \langle m \rangle \times K \cong \mathbb{Z}_2 \times \mathbb{Z}_2$.
\end{enumerate}
\end{example}

\vspace{1.5em}
\begin{corollary}[of proposition in the proof of Theorem 2.69]
If $|G| = p^a$, abelian, then $G \cong \mathbb{Z}_{p^{b_1}} \times \mathbb{Z}_{p^{b_2}} \times \dots \times \mathbb{Z}_{p^{b_\ell}}$, where $\sum_{j=1}^{\ell} b_j = a$. \\
(Proof): Induction on $|G| = p^a$.
\end{corollary}
\vspace{1em}
\begin{example}
How does an abelian $G$ with $|G| = 360 = 2^3 \times 3^2 \times 5$ look like? \\
\textbf{By Step 1:} $G \cong G_8 \times G_9 \times G_5$ \quad where $|G_i| = i$ for $i=5, 8, 9$. \\
\textbf{By Step 2:} $G_8$ can be $\mathbb{Z}_8$ or $\mathbb{Z}_2 \times \mathbb{Z}_4 \cong \mathbb{Z}_4 \times \mathbb{Z}_2$ or $\mathbb{Z}_2 \times \mathbb{Z}_2 \times \mathbb{Z}_2$. \\
\phantom{\textbf{By Step 2:}} $G_9$ can be $\mathbb{Z}_9$ or $\mathbb{Z}_3 \times \mathbb{Z}_3$. \\
\phantom{\textbf{By Step 2:}} $G_5$ can be $\mathbb{Z}_5$ only.

\noindent $\therefore G \cong \begin{pmatrix} \mathbb{Z}_8 \\ \mathbb{Z}_4 \times \mathbb{Z}_2 \\ \mathbb{Z}_2 \times \mathbb{Z}_2 \times \mathbb{Z}_2 \end{pmatrix} \times \begin{pmatrix} \mathbb{Z}_9 \\ \mathbb{Z}_3 \times \mathbb{Z}_3 \end{pmatrix} \times \mathbb{Z}_5$ \quad (with 6 possibilities)
\end{example}
\vspace{2em}
\begin{corollary}[Smith Normal Form]
All finite abelian group $G$ is isomorphic to $G \cong \mathbb{Z}_{d_1} \times \dots \times \mathbb{Z}_{d_k}$ where $d_i > 1$ integers, and $d_i \mid d_{i+1}$ \quad $\forall i=1, \dots, k-1$.
\end{corollary}

\noindent Moreover, if $H \cong \mathbb{Z}_{e_1} \times \dots \times \mathbb{Z}_{e_\ell}$ for $e_i>1$ integers, $e_i \mid e_{i+1}$ $\forall i=1, \dots, \ell-1$. Then: \\
$(G \cong H) \iff (\ell = k \text{ and } d_i = e_i \ \forall i)$